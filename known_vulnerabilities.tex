%\chapter{Known Vulnerabilities}
%\label{chap:known_vulnerabilities}
\section{Security research}

While consumers might see NFC applications as a gadget that will make their lives easier, this development towards contactless payment systems has raised questions from security researchers.
%TODO Summarize and reference some security research done on RFID.

Because NFC will be used in payment and access control, it should be assumed that attackers will try to exploit this technology. Research has been done on the security of NFC and the results show that attacks are possible, which will be discussed below.

Because NFC will be used in payment and access control, it should be assumed that attackers will try to exploit this technology for their own gain. 
%Research has been done on the security of NFC in several areas, among which:

% TODO Referenties toevoegen
%\begin{itemize}
%\item Creating secure off-line payment applications \cite{1592613}
%\item Trusted computing using mobile applications which are managed remotely
%\item Network attacks against \textit{Wireless Personal Area} (WPAN) \textit{Networks} (e.g \textit{Denial of Service}, \textit{Snooping}, \textit{Man-in-the-middle}, etc.) \footnote{Even though NFC isn't strictly a WPAN system as it only allows for 2 communicating parties, this paper also covers network attacks against NFC.}  \cite{1506342}
%\item Intrusion detection mechanisms for WPANs \cite{1361512}
%\item Mitigations agains privacy issues related to wireless payment systems \cite{1527027}
%\end{itemize}

\subsection{Relay attacks}
% TODO
% Maar de NFC telefoon heeft toetsenbord (om actieve approval vd eigenaar te vragen (en te zeggen hoeveel er betaald gaat worden)
RFID, a technology that NFC is compatible with, is prone to relay attacks. A relay attack is a combination of a man-in-the-middle and relay attack. The goal is that the attacker will gain something, but victim will lose something, e.g. the attackers will get an item from a shop, but the victim is the one who is paying. 
To achieve this, the attacker will need a device that fakes a NFC phone or plastic NFC card (ghost) and a device that fakes a reader (leech). The ghost will be used to communicate with a genuine reader and the leech will communicate with a genuine card. The attacker must trick the user into making a legitimate purchase, and have the devices set up in such a way that they operate on much larger distance then the normal operational distance of 10 cm, because then the attack will go unnoticed. \cite{1128470}
With this attack it will also be possible to gain access to the public transport. The check-in gate will communicate with the ghost, which will pass it on to the leech, which will pass it on victim and vice versa. The check-in gate will "think" it is communicating with the victim, but because the attacker is closer to the gate then the victim, he will gain access first and the victim will pay.

\subsection{Countermeasures relay attacks}
There are three possible countermeasures, a Fareday-cage, activation by the user or Two-Factor authentication. A Fareday-cage will block all the wireless communication, so an attacker can make contact with the device. This is not a feasible countermeasure for a mobile phone, because it will not have any connectivity.
With activation of the user, the user will have to switch on NFC on the device, to be able to e.g. make a payment or use public transport. It will not be possible for an attacker to use the phone at any time. This will slighty decrease the user friendliness, because the user will have to switch NFC on and off when it will be used.
With Two-Factor authentication, the user will be asked to present something, e.g. a password, pincode or token. NFC will be turned on all the time, but only when it will be used, the user will type the pincode and a connection is made. The attacker will actually notify the user, that a connection will be made. If the user is aware that the NFC connection is not necessary (because he is not near a payment terminal) and declines or does not type the pincode, the attack is avoided. This countermeassure will also decrease the user friendliness. \cite{1128470}

Overall this attack will not be feasible, because the attacker has to be in relative close range, which will look suspicious. It will also be difficult to maintain an active connection with the terminal and the victim.


%plaatje relay attack

% 2 verschillende scenario's relay attack
% I - stiekem met iemand z'n kaart praten
% II - een frauduleus/gemhackte terminal neerzetten waar iemand - willens en wetens - z'n kaart tegenaan houdt.

\subsection{Malware distribution vectors}
One way to distribute malware is to use the smart poster. With a modified NFC-tag is has been proven to mislead a user to a malicious website. By using space, tab and new line characters in the NFC-tag, the attacker is able to modify the title of the message so that it looks normal (with a title and URL) to the user. The actual URL part of message is not shown to user, but it is still executed. The user is unaware, that he is directed to a malicious website, from there a number of attacks are possible. In the same way it is possible to let the user call or sent a text-message to a different number, e.g. an 0900-number.

It is also possible to hide and spread a worm in this manner. Therefore, the phone of the user has to be infected with the worm. When the phone reads a tag, it will try to change it, like discribed above. The next user will be pointed to a website where he will be tricked into downloading and installing a copy of the worm.

A denial of service attack is also possible, with the goal to let users stop using the service. By using a tag that sends out malformed messages, taped on top of the original tag, the phone of the user will crash when he attempts to read the tag. \cite{10.1109/ARES.2009.46} 

\textbf{TODO} cite 'is your cat infected with a computer virus' paper.

\subsection{Countermeasures malware distribution vectors}


\subsection{Differential Power Analysis Attack}

A Differential Power Analysis (DPA) attack is a form of side channel attack. A side channel attack is based on information that can be collected by analysing the physical implementation of a cryptographical system (cryptosystem) e.g. power usage, electromagnetic leakage or timing. It is possible to prevent this from happening, by implementing extra shielding, filtering inputs and outputs. This is a problem for devices where the cryptosystem has to be small, cheap and low on power usage e.g. the secure element in a mobile device.

With DPA the power consumption is measured to extract information about the workings of the cryptosystem. To measure the power consumption an analog to digital converter is used. When the cryptosystem is processing, the power usage will vary during the different operations. The information is statistically analyzed to determine the key or data, or it can give information about the key, which can help to speed up the attack or help a different attack to succeed. This technique can also be applied for electromagnetic radiation. Information can also be collected from signals generated by data travelling to and from the secure element and memory.
All these variations in ICC, wires and electromagnetic radiation can provide information to be analyzed to yield some information about the key or secret data.

One measurement of these variations might not be enough to produce the entire key, because of noise (e.g. inaccuries, outside influence) that is collected during the measurement, which will provide no information about the key. Therefore several measurements are necessary to distinguish the signal from the noise. By aligning the meassurements, an attacker can compare data on single points of interest. By averaging the collected data, the usable signal is amplified and the noise is filtered out.

Another way to collect information about the key, besides taking many different measurements, are changing some bits of the key to check if the changes might reveal some information about process. This technique can also be repeated and statiscally analyzed to get the key. Also by resetting or switching of the power during a specific operation, which can be repeated, information about the key might be collected.




\subsection{Countermeasures Differential Power Analysis}

The adding of noise to the cryptosystem
Adding some randomness to a crypto operation.

% ERIk open platform vs gesloten platform

%TODO Countermeasures
