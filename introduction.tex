\chapter*{Introduction}
\addcontentsline{toc}{chapter}{Introduction}
\newglossaryentry{PKI}{name={Public Key Infrastructure},description={A cryptographic methode for encrypting messages over untrusted networks.}}
\glsadd{PKI}

\section*{Purpose}

%\section{Introduction} %(beetje dubbel)
Near Field Communication (NFC)  is an upcoming technology and will add a new way of communication on mobile devices.
It will allow a mobile device, e.g. a smartphone, to communicate wirelessly over a short distance with other devices or some form of terminal. There are two major incentives for development, one is mobile payment and the other is ticketing. For example, a user will be able to pay for something using his smartphone, but also gain access to the public transport. 
% (wat algemene technische info over NFC, based on RFID, 13.
% 56 MHz, 106-424 kbps transfer speed, etc)

Recently a few pilots have started in the Netherlands which allow users to pay with their mobile device and gain access…(misschien hier het voorbeeld noemen van OV chipkaart op de telefoon van Erik Poll).
 
For users it will increase ease of use, because he or she will basically have his or her wallet or public transport ticket on his smartphone.
While user sees this as new features or gadget, it also raises a lot of questions on the security perspective. 
Nowadays, when a new technology is made use of by many people, criminals will try to exploit weaknesses in this technology for their own benefit.
Because NFC will be used in payment and ticketing, we assume that people will try to exploit this technology for their own gain. 
We know some research has been done on the security and architecture of NFC  (bronnen en wat uitleg over de inhoud daarvan), but there is no succinct, definitive overview of the security architecture and the known and vulnerabilities.

We will present an overview of the security architecture of NFC applications for mobile devices.
In addition to this we hope to be able to identify specific applications in which we foresee vulnerabilities.
In section ... we will discuss an overview of the hardware, software and communication architecture of a NFC-enabled mobile device and some known vulnerabilities in NFC or NFC-like applications.
 
\section{...}
Further we will look into the possibilities that NFC offers.
 
\section{...}
Further we will look into the possibilities that NFC offers.
Then we will explain how known vulnerabilities work and based on the architecture and these vulnerabilities, we will try to present a number of foreseen vulnerabilities and explain some possible countermeasures against these foreseen vulnerabilities.


%Bronnen:
%IEEE artikelen:
%Near Field Communication Network Services
%Vulnerability Analysis and Attacks on NFC-enabled Mobile Phones
