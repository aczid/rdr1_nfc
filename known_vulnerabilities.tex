%\chapter{Known Vulnerabilities}
\label{chap:known_vulnerabilities}

\section{Security research}

While consumers might see these applications as a gadget that will make their lives easier, this development towards contactless payment systems rightfully raises questions from a security perspective.
%TODO Summarize and reference some security research done on RFID.
Because NFC will be used in payment and access control, it should be assumed that attackers will try to exploit this technology for their own gain. 
Research has been done on the security of NFC in several areas, among which:

% TODO Referenties toevoegen
%\begin{itemize}
%\item Creating secure off-line payment applications \cite{1592613}
%\item Trusted computing using mobile applications which are managed remotely
%\item Network attacks against \textit{Wireless Personal Area} (WPAN) \textit{Networks} (e.g \textit{Denial of Service}, \textit{Snooping}, \textit{Man-in-the-middle}, etc.) \footnote{Even though NFC isn't strictly a WPAN system as it only allows for 2 communicating parties, this paper also covers network attacks against NFC.}  \cite{1506342}
%\item Intrusion detection mechanisms for WPANs \cite{1361512}
%\item Mitigations agains privacy issues related to wireless payment systems \cite{1527027}
%\end{itemize}

\subsection{Relay attacks}
% TODO
% Maar de NFC telefoon heeft toetsenbord (om actieve approval vd eigenaar te vragen (en te zeggen hoeveel er betaald gaat worden)

In \cite{1128470} the working of a relay attack on  a RFID system are explained.
This is also interesting for NFC, because it resembles RFID.
In the article, an attacker will have a device that fakes a card (ghost) (e.g. bank card) and a device that fakes a reader (leech). 
The ghost will be used to communicate with a genuine reader and the leech will communicate with a genuine card.
The trick is, that the attacker will try to pay for something, but the victim is the one paying.
The attacker must trick the user into making a legitimate purchase, and have the devices set up in such a way that they operate on much larger distance then the normal operational distance of 10 cm, because then the attack will go unnoticed.

%plaatje relay attack

% 2 verschillende scenario's relay attack
% I - stiekem met iemand z'n kaart praten
% II - een frauduleus/gemhackte terminal neerzetten waar iemand - willens en wetens - z'n kaart tegenaan houdt.

\subsection{Malware distribution vectors}
In \cite{10.1109/ARES.2009.46} a few attacks with Smart Posters are explained.
% Smart poster hoort bij NFC applications (chapter 2) genoemd te worden
Here the information in NFC-tag will direct the mobile device to a malicious site where the user is tricked into making, for example a financial transaction.
\textbf{TODO} cite 'is your cat infected with a computer virus' paper.

% ERIk open platform vs gesloten platform


