\section{System Architecture}
\label{chap:hardware_architecture}
Strongly related to the progression of RFID applications towards 'electronic cash' is the necessity to verify the authenticity of the input from the user, requiring a mechanism to run trusted code on a user's device.
This poses several problems as storage on and operation of mobile devices are liable to tampering by the user.

%NFC, being grounded in RFID technology, provides some security capabilities previously unavailable in mobile devices.
%These capabilities require storage of sensitive data and code, for which a \textit{Secure element} is used.
\subsection{GSM Security}
The GSM standard allows for secure communication on the network by means of a secret key stored on an \textit{Universal Integrated Circuit Card} (UICC) by the \textit{Mobile Network Operator} (MNO).
Besides storing the secret key, the UICC is running a \textit{Subscriber Identify Module} (SIM) application, which holds the routines used for encryption in GSM, so that the secret key never needs to leave the SIM card.
In fact, the SIM card remains property of the MNO after distribution, much like a bank card or passport remain the property of their respective issuers.
However, currently there is no infrastructure in place for third parties to make use of such secure storage for their own applications.
%These third parties need to be elevated to the role of \textit{Trusted Service Manger}
%MARK: In de tekst hieronder moet duidelijk worden, dat NFC eist/vraagt dat een UICC wordt opgedeeld voor verschillende applicaties

% ERIK: SIM is een voorbeeld van een "Secure Element". Een Secure Element kan ook een andere smartcard chip in de telefoon zijn.

% In de telefoons die 'we' hebben is de SE geen onderdeel van de SIM, maar een los (al dan niet trusted) component.
%MARK: zullen we dit verder uitbreiden in het architectuur hoofdstuk en dit hier niet noemen?


%ERIK Waarom is de architectuur van GSM zoals-ie is & hoe breidt NFC dat uit? Er zijn tig opties van NFC telefoons, dus we kunnen niet overal diep op ingaan.

\subsection{Secure elements}
In \cite{Reveilhac:2009:PSE:1548884.1549404} four different secure elements are proposed, of which the integration on a regular SIM is suggested as the most likely candidate to be used.
%The secure element to be used is a \textit{Universal Integrated Circuit Card} (UICC).
%This is a smartcard which is used in a mobile phone to connect to the GSM and UTMS network. % UICC toevoegen aan glossary, Uitleggen dat het SIM is
UICC is suggested because it's secured, already in use and thus tried, removable and standardized, which means a user can switch phones. 


%TODO Beschrijf eerst bestaande telefoon architectuur (zonder NFC) en de mogelijke architecturen -met- NFC

%ERIK: figuren toevoegen van verschillende telefoon architecturen. 

% SIM ---> telefoon		SIM = secure, tamper-resitent, authenticatie van telefoon aan het netwerk	marketing redenen (onderscheid welke onderdelen van wie zijn) SIM is van de telco
% a SIM ---> telefoon ---> SE 	SmartMX contactless smartcard  (nokia)
% b SD kaart als a maar vervangbaar SE
% c NFC-SIM

The problem of a malicious user can be mitigated by making use of a \textit{Secure Element} (SE) for storing sensitive information.
Because of its sheer microscopic scale, it's very difficult and costly for an attacker to tamper with the function of this device.

Below we will discuss a number of different ways such a secure element can be implemented on a mobile device. 

%\subsubsection{Integrated Secure Element}
The most obvious solution would be to integrate the SE in the device hardware directly.
This works for the intended cryptographic use, but limits the device to communicate with the services aware of its pre-programmed secret key, unless it is capable of securely updating itself with news keys.
%ZZ
Nokia released a phone (Nokia 6131 NFC) in 2006 with NFC as a feature.
The architecture of this phone consists of a SIM card, antenna and a separate Secure Element.
The same layout of this architecture can also be used in other mobile devices, see figure \ref{fig:integrated_se}.
The phone hardware consists of the usual (processor, memory), but also hardware to provide for NFC features and a secure element to enable payment and ticketing.
The secure element stores sensitive data and enables tag and smart card emulation.
In the case of the Nokia phone it is divided in two subcomponents, a Java Card, used for payment, and Mifare 4k, used for ticketing.

%security problems for this architecture:

%\subsubsection{Modular Secure Element}
Another option is an external SD-card which houses all the hardware needed to enable NFC and also the secure element, see figure \ref{fig:modular_se}.
%( http://www.nearfieldcommunicationsworld.com/2009/01/12/3485/tyfone-puts-nfc-into-microsd-cards/)
In this configuration a third party can issue an SD card to its customers which houses a secret key hidden from the user.

%security problems for this architecture: SD card might get stolen and used by somebody else.
%\subsubsection{Multiple SIM cards}
Related to the above architecture is the one depicted in figure \ref{fig:multi_sim}.
Here the architecture consist of a phone with multiple SIM cards and the usual hardware.
In this case a telecom provider will own one SIM card and third-party service provider will own the other.
This way there no problem with splitting the resources of one SIM card and the trust issue between different companies.

%\subsubsection{Trusted SIM card}
Of course the option of splitting the resources is also possible, if different companies can find an agreement to share the same SIM card.
This is possible, because smartcards in general are becoming more powerful.
This makes two different architectures possible.
The first one, as pictured in figure \ref{fig:sim_se}, uses the SIM card as secure element.
A phone will have the usual hardware, hardware making NFC possible and SIM card with applications from the telecom provider and applications from a company chosen by the user, e.g. a bank for payment and a public transport company. 
The second option resembles a combination of the first one and the one with the SD card architecture.
Here a SIM card has all the hardware needed to make NFC possible and it also acts as a secure element, see \ref{fig:integrated_se}.
Like the first option, the resources of the SIM card will be split among the involved parties.

%hier moet nog een bron voor gevonden worden
%security problems for this architecture: SIM card might get stolen and used by somebody else
%hier moet nog een bron voor gevonden worden
%security problems for this architecture: SIM card might get stolen and used by somebody else


\subsection{Adoption}
In \cite{1497411} the authors investigate the possibility of running trusted code in mobile devices and concluded that while such a system is technically feasible, its widespread adoption is hampered by the certification process payment card companies impose on their payment products.
Payment card companies' current standards dictate that their cards may not be modified after production, which poses a problem as this is exactly what makes NFC an enticing alternative to conventional bank cards.

This poses a problem for the implemention of NFC in general. As seen in (figure xx, still to add) the "demand" for NFC is a circular dependency. If the customer doesn't demand NFC from an issuer (bank, PT), the issuer won't request a TSM to create application, so nothing will be putt on a UICC by the MNO and the customer gets nothing. If one of the parties in this figure, refuses to cooperate, NFC will not launch. What we see now (artikel tweakers en global platform) is that issuers and MNO are working together to make NFC available for customers. The next thing to happen is that mobile phones will get NFC hardware and an implementation of a secure element. With Nokia's 6131 NFC phone a first step was made for developers. But now also Google has announced that they will introduce a NFC phone and some rumours suggest that also Apple is developing a new Iphone with NFC, see \cite{nu_artikel}. If customers like NFC because of the possibilities it will take off and demands for other applications will be made. 


% TODO hoort bij Wireless communication, terwijl 3.2 (Secure Elements) heel ergens anders over gaat
%\subsection{Standardization}
%NFC has been described by NFCIP-1 (Near Field Communication Interface Protocol 1) first on ISO18092, ECMA340 and ETSI %TS102 and also NFCIP-2 defined in ISO 21481, ECMA352 and ETSI TS102 312.
%With NFCIP-2, NFC became compliant with the RFID standards of ISO14443 and ISO15693.


% diagram van NFC communicatie

%TODO RFiD card vs NFC - verschil vd terminal
%                      - live GSM verbinding met de bank
%TODO Online vs Offline
%                      - privacy gevoeligheid

%\section{Wireless communication}
%The distance at which the NFC communication takes place is 10 cm, operates at the 13.56 MHz frequency and it has a transfer speed of 106, 216, 424 kbit/s.
%promising secure element alternatives for NFC technology. */


\subsection{Advantages and Limitations}

% Toetsenbord en display feature (semi-trusted terminal, moeilijker te tamperen)

% Voorbeeld telefoon met gewone sim en NFC SD card adapter
% Bankier kan heir apps op installeren

% Voorbeeld telefoon met meerdere sims

% Vorobeeld SIM en losse SE, met trusted code

% Voorbeeld SIM van KPN, met apps van de bank

%TODO Uitzoeken:
% Rabomobiel - SIM vd rabobank


