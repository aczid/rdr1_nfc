\chapter{Known Vulnerabilities}
\label{chap:known_vulnerabilities}

\section{Relay attacks}
In "Picking Virtual pockets using Relay Attacks on Contactless Smartcard Systems" the working of a relay attack on  a RFID system are explained.
This is also interesting for NFC, because it resembles RFID.
In the article, an attacker will have a device that fakes a card (ghost) (e.g. bank card) and a device that fakes a reader (leech). 
The ghost will be used to communicate with a genuine reader and the leech will communicate with a genuine card.
The trick is, that the attacker will try to pay for something, but the victim is the one paying.
The attacker must have the devices set up in such a way that they operate on much larger distance then the normal operational distance of 10 cm, because then the attack will go unnoticed.

%plaatje relay attack

\section{Malware distribution vectors}
In "Vulnerability analysis and attacks on NFC-enabled Mobile phones" a few attacks with Smart Posters (posters with NFC-tags that have some information) are explained.
Here the information in NFC-tag will direct the mobile device to a malicious site where the user is tricked into making, for example a financial transaction.


