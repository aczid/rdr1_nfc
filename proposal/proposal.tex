% !TEX TS-program = pdflatex
% !TEX encoding = UTF-8 Unicode

% This is a simple template for a LaTeX document using the "article" class.
% See "book", "report", "letter" for other types of document.

\documentclass[11pt]{article} % use larger type; default would be 10pt

\usepackage[utf8]{inputenc} % set input encoding (not needed with XeLaTeX)

%%% PAGE DIMENSIONS
\usepackage{geometry} % to change the page dimensions
\geometry{a4paper} % or letterpaper (US) or a5paper or....

\usepackage{graphicx} % support the \includegraphics command and options

%%% PACKAGES
\usepackage{booktabs} % for much better looking tables
\usepackage{array} % for better arrays (eg matrices) in maths
\usepackage{paralist} % very flexible & customisable lists (eg. enumerate/itemize, etc.)
\usepackage{verbatim} % adds environment for commenting out blocks of text & for better verbatim
\usepackage{subfig} % make it possible to include more than one captioned figure/table in a single float
% These packages are all incorporated in the memoir class to one degree or another...

%%% HEADERS & FOOTERS
\usepackage{fancyhdr} % This should be set AFTER setting up the page geometry
\pagestyle{fancy} % options: empty , plain , fancy
\renewcommand{\headrulewidth}{0pt} % customise the layout...
%\lhead{}\chead{}\rhead{}
%\lfoot{}\cfoot{\thepage}\rfoot{}

%%% SECTION TITLE APPEARANCE
%%%\usepackage{sectsty}
%%%\allsectionsfont{\sffamily\mdseries\upshape} % (See the fntguide.pdf for font help)
% (This matches ConTeXt defaults)

%%% ToC (table of contents) APPEARANCE
\usepackage[nottoc,notlof,notlot]{tocbibind} % Put the bibliography in the ToC
\usepackage[titles,subfigure]{tocloft} % Alter the style of the Table of Contents
\renewcommand{\cftsecfont}{\rmfamily\mdseries\upshape}
\renewcommand{\cftsecpagefont}{\rmfamily\mdseries\upshape} % No bold!

%%% END Article customizations

%%% The "real" document content comes below...

\title{Security of NFC telephones}
\author{Mark Vijfvinkel \& Aram Verstegen \\ 4077148 4092368}
\date{} % Activate to display a given date or no date (if empty),
         % otherwise the current date is printed 

\begin{document}
\maketitle

\abstract{

Finding bugs in large pieces of software is traditionally done by testing a finite set of input data. This is both time-consuming and inaccurate, because not all execution paths can be tested in a reasonable amount of time. A recent development is the application of formal methods while developing software to reduce the amount of bugs introduced to the software. ESC/Java2 is one of these applications; it is a tool that preforms formal static verification of Java code annotated with JML. This literary study into ESC/Java2 will look at how successful the application of ESC/Java2 is in the verification of Java source code by looking at several case studies and properties of ESC/Java2.

}

\section{Problem}

These days most people have at least two things in their pocket, a mobile device (this can be a mobile phone, PDA or now a days a combination of those two, called a smartphone) and a wallet. In the wallet there is a certain amount of money, either in the form of cash or bankcard, or both. And with this everything is payed, from small micropayments (candy from a machine) to big purchases in a shop.
\\ Recently a few pilots have been started \textbf { [bron noemen]} to combine the wallet and the mobile device into one thing. This is done by taking a mobile device and adding a NFC implementation. NFC stands for Near Field Communication, which allows a device to communicate wirelessly which an other device. A result of this will be several implementations, for example one implementation will allow people to pay with their mobile device and an other will use tickets for publictransport and events. 
\\ While these applications will bring a lot of possibillities, it will also introduce security risks. In our paper we will focus on the known and forseen vulnerabilities of NFC, by answering the following main research question.

\subsection{Research question}
\begin{center}
What are the known and forseen vulnerabilities of NFC applications for mobile devices? % Moet nog aangepast
\end{center}
\subsubsection{Subquestions}
In order to answer the main research question, we'll answer the following sub-questions.
\begin{center}


What is the architecture of NFC-enabled mobile phones?
\end{center}
\begin{center}
What are some of the known vulnerabilities in NFC or NFC-like applications?
\end{center}
\begin{center}
Profit?
\end{center}

\section{Scope}

\section{Strategy}
More text.

\end{document}
