\section{Introduction}
%\newglossaryentry{PKI}{name={Public Key Infrastructure},description={A cryptographic methode for encrypting messages over untrusted networks.}}
%\glsadd{PKI}

%\section*{Purpose}

\subsection{Near Field Communication} %(beetje dubbel)
Near Field Communication (NFC) is an upcoming wireless communication technology which will allow for new applications on mobile devices, e.g. smartphones.
A smartphone is a mobile phone which has other functionalities besides calling and text-messaging like internet connectivity, email, agenda and the possibility to install third-party software.

NFC is an extension of the \textit{Radio Frequency Identification} (RFID) protocol.
%TODO misschien nog ff RFID algemeen uitleggen (wat een tag/reader is)
In a basic RFID system, the tag consists of a integrated circuit which stores and processes data (e.g. name, fingerprint or product number) and can transmit and receive information via a modulated radio frequency (RF) signal through an antenna.
RFID tags can be passive, semi-active or active.
Active tags contain a battery and transmit at a regular interval whereas semi-active can only send data after they have been activated by reader.
Passive tags are powered through an electromagnetic field created by the reader.
Passive tags do not have batteries and can only communicate with a reader when it is within range (ten centimeters according to specifications, although ranges up to ten meters can be covered by more powerful, specialistic equipent).
Active tags can be read at a range up to about 100 meters.
 
Because NFC is an extension of the RFID protocol, it allows mobile devices to assume the capabilities of an RFID tag or reader, or utilize an RF connection interactively.
By interactive usage, we mean that both parties act as both the sender as well as the receiver.
This will allow a mobile device to communicate wirelessly  with other NFC-capable devices such as the user's own RFID cards, another smartphone, a payment terminal, a so-called 'smart poster' or an access control terminal.
Having NFC functionality available in a mobile handset would allow for use cases including (micro)payments, electronic banking, access control, exchange of digital coupons or interactive advertising.

\textit{Wireless Personal Area Network} (WPAN) technology is already widely available in mobile devices in the from of Bluetooth.
%This technology has a range of ten meters, a transfer speed of 24 Mbit/s (in version 3.0), and is used to transfer data between devices at a relatively short distance.
Bluetooth uses an interactive login to setup a connection between devices, which means that both parties have to type in a key.
Establishing a Bluetooth connection between devices (a process called 'pairing' in Bluetooth terminology) can be performed with lower cognitive load in NFC applications by relying on the close proximity required to setup a connection.
Using this principle of required proximity, NFC can be used to 'bootstrap' a Bluetooth pairing between devices.
Such ease of use is preferable from both a usability and business perspective, but for most of the intended use cases this principle is not enough to ensure security, and some actual security measures are required.
NFC systems require security features similar to those relied upon in RFID technology in order to qualify as a trusted platform which meets the security criteria for payment and access control systems.
A hardware component which provides these security features is called a \textit{Secure Element} (SE).

%There are two major use cases that sparked its development, namely mobile payments and access control. %MARK we zouden access control toch niet meer gebruiken?

%For users of a mobile device NFC will increase the ease of use, because the user will be able to use a mobile device for paying (e.g. at a shop or candy machine) and use the same device to pay for public transport.
%Theoretically the user doesn't have to carry a wallet anymore, only the mobile device.
%For example, instead of using coins to pay for candy, you'll just wave your smartphone in front of the machine, you'll pay for the candy and the money will be taken out of your bank account. 

%plaatje van een NFC betaling

%TODO Telco's en gebruikers er bij
%TODO Manufacturers of what? - telefoons, NFC tags, ???
%\section{Stakeholders} % was dit niet overbodig om te noemen? MARK: volgens mij zei erik of alles noemen of eruit?
%NFC will be exploited by different stakeholders of the NFC Forum, which was started in 2004 to advance NFC technology.
%Stakeholders are hardware component manufacturers, mobile device assembling companies, SIM card producers, application developers and financial service institutions.
%And of course, let's not forget the user.
%Currently the Forum has 140 members, who all work together to promote the use of NFC. 
%Recently a few pilots have started in the Netherlands which allow users to pay with their mobile device and gain access.
%(misschien hier het voorbeeld noemen van OV chipkaart op de telefoon van Erik Poll).

%\subsection{Implementation}
%This allows for new applications to be built on top of existing RFID technology, of which payment and public transport applications are the most important.

\subsection{Motivation}
RFID-based applications for access control and public transport payment are already in place, and are likely to make the transition to NFC-enabled mobile phones.
In our opinion, this transition warrants some serious security research.
A lot of literature on the subject is already available.

However, to our knowledge there has been no succinct, definitive overview leading from the system organisation to the security architecture and the known vulnerabilities of NFC systems.


% "Did you google scholar around for this?"
%The goal of this paper, is to provide the reader with an introduction to NFC security-related research currently ongoing.
%In addition to this we hope to be able to identify specific applications in which we foresee vulnerabilities.
% "Introduction to NFC security _research_"
% vs "Introduction to NFC security _issues/architecture_"
% Naar welk aspect kijken wij? - Verderop in 1.5 geven jullie hier beter antwoord op

%TODO
\textbf{TODO nalopen}
In this paper we will describe various applications of NFC technology and show how these relate to different NFC architectures.
We will also summarize some of the known vulnerabilities surrounding NFC-applications and investigate what possible countermeasures there might be.
We hypothesize that most new vulnerabilities will be found in applications built on top of NFC systems, rather than in the NFC architecture itself.
We envision vendors might try to apply existing business rules to this new technology while they do not fully understand the risks associated with it, which might result in inadequate security measures being taken.

%This is a placeholder for an overview of the subjects covered in our paper, such as applications, hardware and known vulnerabilities and why we chose to cover these topics.
%Also, probably our research question and hypothesis should go in here.

%\subsection{Overview}
%%In chapter \ref{chap:applications} we will briefly summarize some of the applications NFC will likely be used for.
%In chapter \ref{chap:hardware_architecture} we will present an overview of the hardware, software and communication architecture of a NFC-enabled mobile device.
%%\section{...}
%%Further we will look into the possibilities that NFC offers.
%In chapter \ref{chap:known_vulnerabilities} we will summarize security issues on vulnerabilities in NFC or NFC-like applications.

%Based on the architecture and these vulnerabilities, we present a number of foreseen vulnerabilities and explain some possible countermeasures against these foreseen vulnerabilities in chapter \ref{chap:foreseen_vulnerabilities}.


%Bronnen:
%IEEE artikelen:
%Near Field Communication Network Services
%Vulnerability Analysis and Attacks on NFC-enabled Mobile Phones
