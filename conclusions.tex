\label{chap:conclusions}
\section{Conclusions}

At the start of this research we had the following research question and hypothesis. 

\begin{quote}
What is the security architecture of NFC applications for mobile devices and what are some of the known and foreseen vulnerabilities?
\end{quote}

\begin{quote}
\texit{We think that most new vulnerabilities will be found in applications built on top of NFC systems, rather than in the NFC architecture itself. We envision vendors might try to apply existing business rules to this new system, even if their understanding of it is oversimplified.}
\end{quote}

During our research we looked at several applications for NFC. The most imported were the payment application and the public transport application. In these two sectors it is likely that NFC will become available for consumers, because pilots have been started in these sectors and a lot of research has been done in these areas.

Chicken egg problem + verschillende nadelen per architectuur + niet duidelijk welke architectuur het wordt.



The known vulnerabilities are applicable to all of the different architectures mentioned above. It does not matter that the secure element is owned by one party or shared by multiple parties. It also does not matter how the secure element is implemented in the architecture of the mobile device. The attacks mentioned in chapter 4 all apply to the different architectures mentioned in chapter 3. 

As mentioned in paragraph 4.5 differential power analysis is a problem for devices where the secure element has to be small, cheap and have low powerusage.
