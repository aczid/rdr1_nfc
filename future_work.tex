\section{Future work}
\label{sec:future_work}

During our literature study we have come across several areas that warrant further research.
One interesting application of NFC is in the healthcare system.
As mentioned it is possible for caretakers to register their work activities with an NFC enabled device, but the list of applications is longer.
%By creating a small sensor that is powered by a battery or, like an RFID passive tag, by the reader, it's possible to implant this in a in a person.
It is possible for RFID devices to be surgically implanted in a patient.
The sensors inside such a medical tag may measure all kinds of information about a patient, like blood pressure, blood sugar level or eye pressure.
It may be possible to equip a pacemaker with NFC, facilitating readouts of diagnostic information like battery status and adjusting the device's configuration without requiring the patient to undergo surgery. 
The security model of such an application should be highly trustworthy as malicious reconfiguration may cost lives.
%For example, it must not be possible for an attacker to switch off a pacemaker when that person is walking down the street. 
We are so far only aware that the feasibility of such applications is being investigated, but feel the need to stress its possible development should be held to rigorous security standards.

During the research we became aware that NFC truly is an upcoming technology.
We learned that NFC will not become available on smartphones until early in the year 2011.
Payment and transport applications are being developed and tested.

It is not guaranteed that NFC will become a success, its success is dependant on the willingness of the parties involved to make a push for the deployment of NFC technology.
If consumers do not request NFC applications, the other stakeholders like banks and public transport companies will not rush developing them.
%This can result in, for example, MNOs being reluctant to share their UICC, so that applications will not be available. 
Another problem that relates to the above, is the possible trust issues between the involved parties regarding the use of a single secure element shared between them.
Banks and MNOs may rightfully be reluctant to share their secure element with another party, because they do not want any of the parties with which they share, to have access to their private application data.
Also, MNOs do not want other parties to use their UICC as a secure element, because it might interfere with the primary function of the device, namely network authentication.
Also marketing plays a role: one party may not want to share the same branded card with a competitor.
%This prevented us from making a found prediction about which architecture will be implemented. This should be reviewed, if NFC gets implemented and used.

In this paper we mainly looked at how NFC is implemented from the client-side point of view.
The back-end systems which execute and process the transactions, e.g. billing for a public transport system, will probably have vulnerabilities as well.
For example, a cross-site scripting or SQL-injection attack might be possible through emulation of malicious RFID tags.\cite{rieback2006your}
The security architectures adopted by these kinds of application attacks appear to be worthy of further research as they are developed.

%While the possibility of secure offline payments using NFC is intriguing, the hardware is not yet powerful enough to accommodate the required cryptographic power to authenticate transaction swiftly.
