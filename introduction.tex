\section{Introduction}
%\newglossaryentry{PKI}{name={Public Key Infrastructure},description={A cryptographic methode for encrypting messages over untrusted networks.}}
%\glsadd{PKI}

%\section*{Purpose}

\subsection{Near Field Communication} %(beetje dubbel)
Near Field Communication (NFC) is an upcoming wireless communication technology which will allow for new applications on mobile devices, e.g. smartphones.
It is an extension of the RFID protocol which allows these devices to assume the capabilities of an RFID card or reader, or utilize an RFID connection interactively.
 
NFC will allow a mobile device to communicate wirelessly over a short distance with other NFC-capable devices such as another smartphone, a payment terminal, a so-called 'smart poster' or an access control terminal.
While \textit{Wireless Personal Area Network} (WPAN) technlogy for mobile devices is already widely deployed in the form of Bluetooth, NFC adds security features also relied upon by RFID which cause it to be considered applicable for security use cases.

%There are two major use cases that sparked its development, namely mobile payments and access control. %MARK we zouden access control toch niet meer gebruiken?

%For users of a mobile device NFC will increase the ease of use, because the user will be able to use a mobile device for paying (e.g. at a shop or candy machine) and use the same device to pay for public transport.
%Theoretically the user doesn't have to carry a wallet anymore, only the mobile device.
%For example, instead of using coins to pay for candy, you'll just wave your smartphone in front of the machine, you'll pay for the candy and the money will be taken out of your bank account. 

%plaatje van een NFC betaling

%TODO Telco's en gebruikers er bij
%TODO Manufacturers of what? - telefoons, NFC tags, ???
%\section{Stakeholders} % was dit niet overbodig om te noemen? MARK: volgens mij zei erik of alles noemen of eruit?
%NFC will be exploited by different stakeholders of the NFC Forum, which was started in 2004 to advance NFC technology.
%Stakeholders are hardware component manufacturers, mobile device assembling companies, SIM card producers, application developers and financial service institutions.
%And of course, let's not forget the user.
%Currently the Forum has 140 members, who all work together to promote the use of NFC. 
%Recently a few pilots have started in the Netherlands which allow users to pay with their mobile device and gain access.
%(misschien hier het voorbeeld noemen van OV chipkaart op de telefoon van Erik Poll).

%\subsection{Implementation}
This allows for new applications to be built on top of existing RFID technology, of which payment and public transport applications are the most important.
RFID-based applications for access control and public transport payment are already in place, and are likely to make the transition to NFC-enabled mobile phones.

\subsection{Motivation}
In our opinion, this trend warrants some serious security research. To our delight, there is already a wealth of literature available on the subject.

However, to our knowledge there has been no succinct, definitive overview leading from the system organisation to the security architecture and the known vulnerabilities of NFC systems.
% "Did you google scholar around for this?"
What we hope to accomplish with this paper is to provide the reader with an introduction to NFC security-related research currently ongoing.
%In addition to this we hope to be able to identify specific applications in which we foresee vulnerabilities.
% "Introduction to NFC security _research_"
% vs "Introduction to NFC security _issues/architecture_"
% Naar welk aspect kijken wij? - Verderop in 1.5 geven jullie hier beter antwoord op

%TODO
\textbf{TODO}
This is a placeholder for an overview of the subjects covered in our paper, such as applications, hardware and known vulnerabilities and why we chose to cover these topics.
Also, probably our research question and hypothesis should go in here.

%\subsection{Overview}
%%In chapter \ref{chap:applications} we will briefly summarize some of the applications NFC will likely be used for.
%In chapter \ref{chap:hardware_architecture} we will present an overview of the hardware, software and communication architecture of a NFC-enabled mobile device.
%%\section{...}
%%Further we will look into the possibilities that NFC offers.
%In chapter \ref{chap:known_vulnerabilities} we will summarize security issues on vulnerabilities in NFC or NFC-like applications.

%Based on the architecture and these vulnerabilities, we present a number of foreseen vulnerabilities and explain some possible countermeasures against these foreseen vulnerabilities in chapter \ref{chap:foreseen_vulnerabilities}.


%Bronnen:
%IEEE artikelen:
%Near Field Communication Network Services
%Vulnerability Analysis and Attacks on NFC-enabled Mobile Phones
