% !TEX TS-program = pdflatex
% !TEX encoding = UTF-8 Unicode

% This is a simple template for a LaTeX document using the "article" class.
% See "book", "report", "letter" for other types of document.

\documentclass[11pt]{article} % use larger type; default would be 10pt

\usepackage[utf8]{inputenc} % set input encoding (not needed with XeLaTeX)
\usepackage{url}

%%% PAGE DIMENSIONS
\usepackage{geometry} % to change the page dimensions
\geometry{a4paper} % or letterpaper (US) or a5paper or....

\usepackage{graphicx} % support the \includegraphics command and options

%%% PACKAGES
\usepackage{booktabs} % for much better looking tables
\usepackage{array} % for better arrays (eg matrices) in maths
\usepackage{paralist} % very flexible & customisable lists (eg. enumerate/itemize, etc.)
\usepackage{verbatim} % adds environment for commenting out blocks of text & for better verbatim
\usepackage{subfig} % make it possible to include more than one captioned figure/table in a single float
% These packages are all incorporated in the memoir class to one degree or another...

%%% HEADERS & FOOTERS
\usepackage{fancyhdr} % This should be set AFTER setting up the page geometry
\pagestyle{fancy} % options: empty , plain , fancy
\renewcommand{\headrulewidth}{0pt} % customise the layout...
%\lhead{}\chead{}\rhead{}
%\lfoot{}\cfoot{\thepage}\rfoot{}

%%% SECTION TITLE APPEARANCE
%%%\usepackage{sectsty}
%%%\allsectionsfont{\sffamily\mdseries\upshape} % (See the fntguide.pdf for font help)
% (This matches ConTeXt defaults)

%%% ToC (table of contents) APPEARANCE
\usepackage[nottoc,notlof,notlot]{tocbibind} % Put the bibliography in the ToC
\usepackage[titles,subfigure]{tocloft} % Alter the style of the Table of Contents
\renewcommand{\cftsecfont}{\rmfamily\mdseries\upshape}
\renewcommand{\cftsecpagefont}{\rmfamily\mdseries\upshape} % No bold!

%%% END Article customizations

%%% The "real" document content comes below...

\title{Security of NFC-enabled mobile devices}
\author{Mark Vijfvinkel \& Aram Verstegen \\ 4077148 4092368}
\date{} % Activate to display a given date or no date (if empty),
         % otherwise the current date is printed 

\begin{document}
\maketitle

\abstract{

Near Field Communication \cite{nfcstd} is a short-range, low-power wireless communication system that uses the same principles as various \textit{RFID} implementations.
It's intended to be integrated into mobile devices, allowing for \textit{contactless} payment applications.
NFC is a relatively new technology which might have a big impact on the way we pay and gain access to buildings, computer systems or even public transport.
Because of these applications and the number of people that will be using it, it is necessary to research how secure NFC really is.
In this literature study we will look at the possibilities of NFC applications for mobile devices, document the architecture of NFC systems and identify some of the known and foreseen vulnerabilities.
}

\section{Problem}
These days most people have at least two things in their pocket, a mobile device (this can be a mobile phone, PDA or nowadays a combination of those two, called a \textit{smartphone}) and a wallet.
In the wallet there is a certain amount of money, either in the form of cash or a bankcard, or both.
And with this everything is paid, from small \textit{micropayments} (e.g. candy from a machine) to big purchases in a shop.

Recently a pilot has started \cite{payter} that integrates a person's payment card(s) into their mobile device using the NFC standard \cite{nfcstd}.

\newpage
\subsection{Research question}

At the end of our research we will try to answer to following research question: 

\begin{quote}
What is the security architecture of NFC applications for mobile devices and what are some of the known and foreseen vulnerabilities? % Moet nog aangepast
\end{quote}

\subsubsection{Sub-questions}

In order to answer the main research question, we'll answer the following sub-questions.

\begin{itemize}
\item [-] What is the architecture of NFC-enabled mobile device? (e.g. hardware, software and communication)

\item [-] What are the possibilities of the NFC devices?

\item [-] What are some of the known vulnerabilities in NFC or NFC-like applications?

\item [-] How do these vulnerabilities work?

\item [-] Based on the architecture and known vulnerabilities, what could some foreseen vulnerabilities be?

\item [-] What might be possible countermeasures against these foreseen vulnerabilities?

\end{itemize}

\subsubsection{Hypothesis}
We are able to give the following hypothesis:

We think that most new vulnerabilities will be found in applications built on top of NFC systems, rather than in the NFC architecture itself.
We envision vendors might try to apply existing business rules to this new system, even if their understanding of it is oversimplified. % Even geen idee hoe ik hier verder kan

\section{Scope}
We are aware that because of the time restrictions in place, we won't be able to cover the entire breadth and depth of NFC systems.
In some implementations of an NFC system, a back-end is used to handle transactions.
Because the back-end is likely to be different for each implementation, we will scope our research down to the technical details of a generalized NFC interaction.
We will provide an up-to-date introduction to the possibilities of an NFC-enabled mobile device and the architecture it's built upon, focussing on the security model.
%focusing mainly on security-related components.
Using this work as a guideline we will document known vulnerabilities in NFC applications, examining their technical details.
We will mention known countermeasures for any vulnerabilities that we discuss.
If time permits, we hope to illustrate these vulnerabilities by including a case study. %or explore the possibilities of the NFC architecture.

\section{Motivation}

With the current technology it is possible to make everyday jobs or actions easier and more comfortable.
It is now possible to take your phone with you, instead of calling from a house, as was the case a few years ago.
Instead of paying with coins and bills, we now pay using a bankcard.
But in this century, the evolution of technology does not stop. So applications like paying with your phone come to mind.
These days the development schedules are so tight, some details may be overlooked because companies may simply try to apply old methods to new technology.
For systems requiring a high level of security, these kinds of oversights can be devastating.
%In IT this is mostly security related. 
\\

\noindent In our opinion security issues should be addressed in any ICT project.
Especially when the system will be exposed to the general public and a lot of people will depend on it for their day-to-day lives.
When NFC-enabled devices take off and people start using NFC-based applications, it will be too late to makes significant changes that will improve security.
Therefore we think it's very important to research and document the known security issues regarding NFC-enabled devices.

\section{Strategy}
In our study we want to investigate the architecture of NFC-enabled devices with a focus on the technical details of the security model.
We first hope to reach a general, high-level understanding of the system's capabilities and uses, and write a solid introduction before delving into the more technical details.
%We hope to present a clear introduction on the state of NFC systems, citing some notable deployments and prospective uses.
%After this groundwork, we will focus on the technical details of the NFC-enabled devices highlighting some of the security vulnerabilities and risks involved.
We will do this by summarizing existing work (such as \cite{Paus2007}, \cite{1731794} and others) on the general workings of NFC systems.

After this high-level introduction which illustrates the context of our research we will limit our scope to the NFC devices themselves and will document the architecture of the security mechanisms in place.
There is plenty of literature (such as \cite{mulliner09:vulnanamms}, \cite{Kfir05pickingvirtual} and others) to guide us in researching the security aspects of NFC systems. 
We hope to be able to point out some known or possible vulnerabilities in these.
We would like to also include a case study of an NFC system implementation.
This could be entirely theoretical but if time permits, would warrant some practical experimentation.

\section{Time schedule}
\begin{center}
  \begin{tabular}{ | r | l | l || l | }
    \hline
    Week & Date & Activity & Deliverable \\ \hline
    39 &  & Researching NFC &  \\ \hline
    40 & 8-10 & Researching NFC & Introduction chapter \\ \hline
    41 & 15-10 & Technical analysis, NFC communication & chapter NFC communication \\ \hline
    42 & 22-10 & Technical analysis, hardware architecture & chapter hardware architecture \\ \hline
    43 & & Autumn break & \\ \hline
    44 & 5-11 & Technical analysis, generic software & chapter generic software \\ \hline
    45 & & Case studies, analyse security & \\ \hline
    46 & 19-11 & Case studies, analyse security & chapter case studies\\ \hline
    47 & & Preparing presentation and draft paper & \\ \hline
    48 & & Preparing presentation and draft paper & \\ \hline
    49 & 6-12 & & Draft paper and presentation sheets due \\ \hline
    49 & 7-12 & & Final presentation  \\ \hline
    49 & 9-12 & Supervisor review & \\ \hline
    50 & 17-12 & Feedback due & \\ \hline
    51 & & Christmas break & \\ \hline
    52 & & Christmas break & \\ \hline
    1 & & Revise paper & \\ \hline
    3 & 17-01 & Revise paper & Revised final paper due \\ \hline
  \end{tabular}
\end{center}

\bibliographystyle{alpha}
\bibliography{references}
\end{document}
