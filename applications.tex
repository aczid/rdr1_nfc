\section{Applications}
While NFC is an emerging technology, some areas in which it is likely to be applied can already be identified.
Various parties throughout the areas of public transport, advertising, building security and banking have expressed interest in employing NFC technology.
% TODO ref

For illustrative purposes a few applications in which NFC can be used are mentioned below.

\subsection{Public Transport}
%Therefore, the user doesn't have to go to a terminal, which will improve the user experience.
%Another application for NFC is public transport.
%This way it is possible to use a mobile device to pay and gain access to i.e. a train, bus, or metro.
%By holding your mobile device in front of a terminal you'll pay for a ticket or show your credit and a gate will open allowing you to enter.
A ticket stored inside a smartphone has some advantages over an RFID card, e.g. users can check the balance on the device and increase it immediately by paying online.

Public transport companies in the Netherlands have largely adopted RFID technology for their electronic ticketing system, through infrastructure provider \textit{Trans Link Systems} (TLS).
%TODO referentie naar pilot
TLS was created as a joint venture between Dutch public transport companies to realise an electronic payment system for their services.
They have also been investigating the feasibility of mobile phone-based payment applications using NFC. \cite{cipit_artikel}

The use of NFC technology for electronic ticketing in public transport is currently considered unfeasible by TLS because it would require users to own an NFC-enabled handset; devices which so far have no significant market penetration in the Netherlands.
Also, the relatively high cost of an NFC-enabled phone as well as the requirement to keep the device powered on during the length of a trip are aspects of NFC which go against TLS's aim for a low threshold in using their electronic ticket system.
%It would also prohibitively expensive for most users and NFC phones are not widely available.

Support for NFC-enabled devices in TLS's system could improve user experience by giving users the option to check or increase the balance of their electronic ticket account at any given time, alleviating the often heard complaint of currently only being able to do so at (most) train stations.
TLS are still investigating the possibility of adopting NFC technology as an alternative payment method in the future. \cite{OVchipkaart} %referentie naar TLS

%hasn't been implemented on a mobile device yet, but plans are being made.
%MARK niet zo lekker stukje kan een bron bij, maar wat voor IEEE, ACM, scholar? Nog geen pilot in nederland :(


\subsection{Payment}
In the summer of 2004, Japanese telecommunications provider \textit{NTT DoCoMo} initiated wide deployment of RFID technology for payment applications in Japan, where now virtually every issuer of payment cards supports its trademark \textit{Osaifu-Keitai} (`mobile wallet') system.
Osaifu-Keitai is based on the \textit{FeliCa} RFID chip developed by \textit{Sony}, and is interoperable with \textit{FeliCa} RFID applications such as the \textit{Suica} public transport card and the \textit{Edy} electronic payment card.
\cite{yamauchi2006intensive}

%This de-facto standard for mobile payment has been integrated into the mobile Suica service, where the mobile device emulates the behaviour of an RFID card. 
%In the future more advanced applications could be created by making use of the active communication possibilities of NFC. 

With the ability to directly transfer data between two parties, NFC technology might come to be used as a medium for peer-to-peer electronic payments.
This concept has been explored in \textit{mFerio}, a proof of concept peer-to-peer mobile payment system which was designed to be as fast and easy to learn and use, and to be as available as cash.
The proposed system should also improve accuracy and speed, while still meeting security criteria like transaction integrity, anonymity, tamper-resistance, impossibility to replicate and theft resilience. 
%\textit{mFerio} uses NFC because of three advantages over other mechanisms, namely it has a short range (harder to intercept transactions), it is quick and easy to set up and a user knows with which device communication is set up.
In \textit{mFerio}, the \textit{Secure Element} (SE) contains the electronic cash and personal details of the user, which by design will not be accessible by unauthorised users if the SE is hardware protected \cite{1555846}.

Another possibility is to create an offline payment system with the use of electronic vouchers.
In this scheme, an \textit{Issuer} can distribute so-called `eVouchers' to \textit{Beneficiaries} via SMS.
These beneficiaries can then then exchange vouchers amongst themselves or transfer the eVoucher to an \textit{Affiliate} payment terminal.
%rom an issuer and transfer this eVoucher to other users (via NFC, SMS or an RFID tag) and payment terminals.
Users are also able to check the balance, history and expiration date of their eVouchers. 

%There are three risks concerning this implementation, first the possibility of copying the eVouchers, second counterfeiting them and third loss of eVouchers in a transaction.
The secure element will only accept software from a Trusted Service Manager, which will have a private key for authentication to the SE.
The SE will store the eVouchers and also encrypt them, if a user decides to sent them to another user. \cite{1592613}

%In \cite{1592613} the authors investigate the technical challenges of a system using cryptographically secure vouchers %which can be exchanged between beneficiaries.
%\textbf{TODO} add their conclusion.
% TODO conclusion
%They concluded the biggest limitation....
%MARK het bovenstaande werd weggelaten of wat was hier ook de bedoeling weer van?

\subsection{Advertisement}
Advertisement is another area in which NFC can be used. By equipping advertisement media (e.g. billboards, flyers, posters) with a NFC/RFID-tag, users can receive the details of a product immediately on their mobile phone.
For example, by supplying the tag with an URL, a user can immediately visit a website and take further action there, e.g. sign-up or fill in a form.
As we will see later on, this can create a security problem for a user that is unaware. \cite{mulliner2009vulnerability}

\subsection{Access Control}
It is also possible to use NFC to gain access to a certain area e.g. a building or room. The company Nedap Healthcare developed an application for caretakers in the home care business. By using an NFC enabled phone, the caretaker can gain access to the house of the client. Upon arrival and leaving the caretaker holds the phone next to the membershipcard of the client, which will register the visit of the caretaker and given care. \cite{Nedap1}, \cite{Nedap2}

\subsection{Bluetooth bootstrapping}
\textit{Wireless Personal Area Network} (WPAN) technology is already widely available in mobile devices in the from of Bluetooth.
%This technology has a range of ten meters, a transfer speed of 24 Mbit/s (in version 3.0), and is used to transfer data between devices at a relatively short distance.
Bluetooth uses an interactive login to setup a connection between devices, which means that both parties have to type in a key.
Establishing a connection between devices (a process called `pairing' in Bluetooth terminology) can be performed with lower cognitive load in NFC applications by relying on the close proximity required to setup a connection.
Using this principle of required proximity, NFC can be used to `bootstrap' a Bluetooth pairing between devices with a higher degree of usability by simply touching the devices and accepting the established connection. \cite{scarfone2008guide}
%Such ease of use is preferable from both a usability and business perspective.
%, but for most of the intended use cases this principle is not enough to ensure security, and some actual security measures are required.

%\subsection{Healthcare}
%Another application for NFC is in the healthcare system.
%In \cite{RFIDHB} an application of NFC is used to measure the pressure in the eye.
%A look at Google Scholar gave us the impression that the use of NFC in the healthcare system, is still in the development stage. %ERIK: niet zo sterk, vervangen door "The current scientific literature gives the impression...", Sommige van deze publicaties noemen.
%Therefore we will leave this application out of our research and recommend it for future work. %Mark niet zo sterk bewijs


%\section{Implementation} %goede titel nodig?
%As mentioned above, there are different NFC applications (of which payment and public transport are the most important) and these applications are starting to be implemented. There are several reasons why NFC hasn't been implemented yet. First of all, the mobile devices that use NFC, still have to be made available to the consumer, but the consumer will only be interested if he or she can actually do something with NFC. So applications are needed, but therefore different parties have to work together. If i.e. a SIM-card is used as a secure element (see Chapter 3), a bank will need to trust the telecom provider and vice versa, but a bank doesn't like to use a, in their eyes, unsafe SIM-card and a telecom provider doesn't want someone else to mess with their SIM-card. Also the costs involved with trying to get all the shops using NFC, are quite large, because new payterminals have to be purchased.
%So all parties have to participate and create agreements to get NFC off the ground, which is what they have been doing, only this takes time. %niet zo sterke zin en gebaseerd op de meeting met erik


% Kip & ei probleem
% Genoeg NFC handsets <-> apps
% trust: telco <-> bank <-> telefoonmaker
% kosten voor payment -> nfc reader bij alle kassa's
% iedereen moet mee doen

% MARK weet niet zo goed wat ik met het onderstaande commentaar moet doen

% SIM - 4k ram 16k rom 64k eeprom

%            trust
% security <             afluister
%            aanvaller < relay attacks
%                        malware

%\section{Pilot programs}
% /* promising secure element alternatives for NFC technology.
%TODO door wie, hoeveel participanten etc is wel nuttig om te weten
%TODO Er wordt er ook maar 1 genoemd

%In London there has been a pilot between November 2007 and May 2008, during which 78 percent participants were interested in using this technique.
%According to networking hardware vendor \textit{Juniper Networks}, 700 million of NFC-enabled mobile phones will be in use by 2013. 
%One of the factors holding back the widespread adoption of NFC, is stakeholders (e.g. banks, mobile providers and public transport companies) have so far been unable to agree on standardization of the required security implements. 

% sorry maar even herschreven om niet meteen heel specifiek te gaan.
%The secure element consists of hardware, software, interfaces and protocols in a mobile device.
%It provides a secure area for storage, running of applications, protection of payments and can be used for authentication  and for applications which need security mechanism.
% dit stukje is niet zo precies, en moet in de introductie minder technisch uitgelegd worden denk ik

%TODO dit moet eigenlijk ergens anders heen
%There are several reasons why NFC hasn't been implemented yet.
%First of all, the mobile devices that use NFC, still have to be made available to the consumer, but the consumer will only be interested if he or she can actually do something with NFC.
%So applications are needed, but therefore different parties have to work together.
%If i.e. a SIM-card is used as a secure element (see Chapter 3), a bank will need to trust the telecom provider and vice versa, but a bank doesn't like to use a, in their eyes, unsafe SIM-card and a telecom provider doesn't want someone else to mess with their SIM-card.
%Also the costs involved with trying to get all the shops using NFC, are quite large, because new payterminals have to be purchased.
%So all parties have to participate and create agreements to get NFC off the ground, which is what they have been doing, only this takes time. %niet zo sterke zin en gebaseerd op de meeting met erik

