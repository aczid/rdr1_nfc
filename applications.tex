\section{Applications}
While NFC is an emerging technology, there are some areas in which it is likely to be applied.
Various parties throughout the areas of public transport, building security and payment card companies have expressed interest in employing NFC technology.

For illustrative purposes a few applications in which NFC can be used are mentioned below.

\subsection{Public Transport}
%Another application for NFC is public transport.
%This way it is possible to use a mobile device to pay and gain access to i.e. a train, bus, or metro.
%By holding your mobile device in front of a terminal you'll pay for a ticket or show your credit and a gate will open allowing you to enter.
Public transport companies in the Netherlands have largely adopted RFID technology for payment and access control, through infrastructure provider \textit{Trans Link Systems} (TLS).
%TODO referentie naar pilot - DONE?
TLS was created as a joint venture between Dutch public transport companies to realise an electronic payment system for their services. They have also been investigating the feasibility of mobile phone-based payment applications using NFC.
The use of NFC technology for payment in public transport is currently considered unfeasible by TLS because it would require users to own an NFC-enabled handset; devices which so far have no significant market penetration in the Netherlands.
Also, the relatively high cost of an NFC-enabled phone as well as the requirement to keep the device powered on during the length of a trip are aspects of NFC which go against TLS's aim for a low threshold in using their electronic ticket system.
%It would also prohibitively expensive for most users and NFC phones are not widely available.
TLS are still investigating the future possibilities of adopting NFC technology as an alternative payment method. \cite{OVchipkaart} %referentie naar TLS
%hasn't been implemented on a mobile device yet, but plans are being made.
A mobile device has advantages over a card, e.g. you can check the balance on your mobile device and increase it immediately by paying online. Therefore, the user doesn't have to go to a terminal, which will increase the user experience.
Support for NFC-enabled devices in TLS's system could improve user experience by giving users the option to check or increase the balance of their electronic ticket account at any given time, alleviating the oft-heard complaint of currently only being able to do so at (most) train stations.


%MARK niet zo lekker stukje kan een bron bij, maar wat voor IEEE, ACM, scholar? Nog geen pilot in nederland :(


\subsection{Payment}
Japanese telecommunications provider \textit{NTT DoCoMo} has initiated wide deployment of NFC technology in Japan, where virtually every issuer of payment cards supports the \textit{Sony}'s \textit{Felica Mobile Wallet} \cite{yamauchi2006intensive}.
In this de-facto standard for mobile payment the mobile device emulates the behaviour of an RFID card, but in the future much more is possible by making use of the active communication possibilities of NFC. For example, by being able to transfer money between two people, NFC technology might become a replacement for cash. E.g. the payment application \textit{mFerio} should be as easy to use and be as available as cash. It should also improve accuracy and speed, while still meeting security criteria like transaction integrity, anonymity, tamper-resistance, impossibility to replicate and theft resilience. 
\textit{mFerio} uses NFC, because of three advantages above other mechanisms, namely it has a short range (harder to intercept transactions), it is quick and easy to set up and a user knows with which device communication is set up. The secure element (SE) in \textit{mFerio} contains cash and personal details of the user, which will not be accessible by criminals if the secure element is hardware protected \cite{1555846}.

A possibility to create an offline payment system, is the use of electronic vouchers. A user can recieve an 'eVoucher' by an SMS from an issuer and transfer the eVoucher to other users (by NFC, SMS or a RFID tag) and payment terminals. Users are also able to check the balance, history and expiration date of the eVouchers. 
There are three risks concerning this implementation, first the possibility of copying the eVouchers, second counterfeiting them and third loss of eVouchers in a transaction. The secure element will only accept software from a Trusted Service Manager, which will have a private key for authentication to the SE. The SE will store the eVouchers and also encrypt them, if a user decides to sent them to another user. \cite{1592613}

%In \cite{1592613} the authors investigate the technical challenges of a system using cryptographically secure vouchers %which can be exchanged between beneficiaries.
%\textbf{TODO} add their conclusion.
% TODO conclusion
%They concluded the biggest limitation....
%MARK het bovenstaande werd weggelaten of wat was hier ook de bedoeling weer van?

\subsection{Advertisement}
Advertisement is another area in which NFC can be used. By equipping advertisements (e.g. billboards, flyers, posters) with a NFC/RFID-tag, users can check the details of a product immediately on their mobile phone. Or by supplying the tag with an URL, a user can immediately visit a website and take further action there, e.g. sign-up or fill in a form. As we will see in chapter 4, this can create a security problem for a user that is unaware. \cite{10.1109/ARES.2009.46}

%\subsection{Healthcare}
%Another application for NFC is in the healthcare system.
%In \cite{RFIDHB} an application of NFC is used to measure the pressure in the eye.
%A look at Google Scholar gave us the impression that the use of NFC in the healthcare system, is still in the development stage. %ERIK: niet zo sterk, vervangen door "The current scientific literature gives the impression...", Sommige van deze publicaties noemen.
%Therefore we will leave this application out of our research and recommend it for future work. %Mark niet zo sterk bewijs


%\section{Implementation} %goede titel nodig?
%As mentioned above, there are different NFC applications (of which payment and public transport are the most important) and these applications are starting to be implemented. There are several reasons why NFC hasn't been implemented yet. First of all, the mobile devices that use NFC, still have to be made available to the consumer, but the consumer will only be interested if he or she can actually do something with NFC. So applications are needed, but therefore different parties have to work together. If i.e. a SIM-card is used as a secure element (see Chapter 3), a bank will need to trust the telecom provider and vice versa, but a bank doesn't like to use a, in their eyes, unsafe SIM-card and a telecom provider doesn't want someone else to mess with their SIM-card. Also the costs involved with trying to get all the shops using NFC, are quite large, because new payterminals have to be purchased.
%So all parties have to participate and create agreements to get NFC off the ground, which is what they have been doing, only this takes time. %niet zo sterke zin en gebaseerd op de meeting met erik


% Kip & ei probleem
% Genoeg NFC handsets <-> apps
% trust: telco <-> bank <-> telefoonmaker
% kosten voor payment -> nfc reader bij alle kassa's
% iedereen moet mee doen

% MARK weet niet zo goed wat ik met het onderstaande commentaar moet doen

% SIM - 4k ram 16k rom 64k eeprom

%            trust
% security <             afluister
%            aanvaller < relay attacks
%                        malware

%\section{Pilot programs}
% /* promising secure element alternatives for NFC technology.
%TODO door wie, hoeveel participanten etc is wel nuttig om te weten
%TODO Er wordt er ook maar 1 genoemd

%In London there has been a pilot between November 2007 and May 2008, during which 78 percent participants were interested in using this technique.
%According to networking hardware vendor \textit{Juniper Networks}, 700 million of NFC-enabled mobile phones will be in use by 2013. 
%One of the factors holding back the widespread adoption of NFC, is stakeholders (e.g. banks, mobile providers and public transport companies) have so far been unable to agree on standardization of the required security implements. 

% sorry maar even herschreven om niet meteen heel specifiek te gaan.
%The secure element consists of hardware, software, interfaces and protocols in a mobile device.
%It provides a secure area for storage, running of applications, protection of payments and can be used for authentication  and for applications which need security mechanism.
% dit stukje is niet zo precies, en moet in de introductie minder technisch uitgelegd worden denk ik

%TODO dit moet eigenlijk ergens anders heen
%There are several reasons why NFC hasn't been implemented yet.
%First of all, the mobile devices that use NFC, still have to be made available to the consumer, but the consumer will only be interested if he or she can actually do something with NFC.
%So applications are needed, but therefore different parties have to work together.
%If i.e. a SIM-card is used as a secure element (see Chapter 3), a bank will need to trust the telecom provider and vice versa, but a bank doesn't like to use a, in their eyes, unsafe SIM-card and a telecom provider doesn't want someone else to mess with their SIM-card.
%Also the costs involved with trying to get all the shops using NFC, are quite large, because new payterminals have to be purchased.
%So all parties have to participate and create agreements to get NFC off the ground, which is what they have been doing, only this takes time. %niet zo sterke zin en gebaseerd op de meeting met erik

