%\chapter{Known Vulnerabilities}
\section{Security research}
\label{sec:known_vulnerabilities}

While consumers might see NFC applications as a gadget that will make their lives easier, this development towards contactless payment systems has raised questions from security researchers.
%TODO Summarize and reference some security research done on RFID.

Because NFC will be used in payment and access control, it should be assumed that attackers will try to exploit this technology.
NFC's security features have been the subject of research and some attacks found possible which will be summarized below.
Because NFC will be used in payment and access control, it should be assumed that attackers will try to exploit this technology for their own gain. 
%Research has been done on the security of NFC in several areas, among which:

% TODO Referenties toevoegen
%\begin{itemize}
%\item Creating secure off-line payment applications \cite{1592613}
%\item Trusted computing using mobile applications which are managed remotely
%\item Network attacks against \textit{Wireless Personal Area} (WPAN) \textit{Networks} (e.g \textit{Denial of Service}, \textit{Snooping}, \textit{Man-in-the-middle}, etc.) \footnote{Even though NFC isn't strictly a WPAN system as it only allows for 2 communicating parties, this paper also covers network attacks against NFC.}  \cite{1506342}
%\item Intrusion detection mechanisms for WPANs \cite{1361512}
%\item Mitigations agains privacy issues related to wireless payment systems \cite{1527027}
%\end{itemize}

\subsection{Relay attacks}
% TODO
% Maar de NFC telefoon heeft toetsenbord (om actieve approval vd eigenaar te vragen (en te zeggen hoeveel er betaald gaat worden)
RFID, a direct ancestor of NFC, is prone to relay attacks.
A relay attack is a combination of a man-in-the-middle\footnote{In a man-in-the-middle attack a malicious attacker places himself or herself between two communicating parties without either party knowing that their communication flows through this man in the middle.} and replay attack\footnote{In a replay attack, the data used for an authentic transaction is intercepted and retransmitted (i.e. `replayed') for malicious purposes}.
The goal is that the attacker will gain something, but the victim will lose something (e.g. the attacker will get a product from a retailer, but the victim is the one who is paying).

To achieve a relay attack, an attacker will need a device that emulates an RFID card (called the \textit{ghost}), and a device that acts as an RFID reader (called the \textit{leech}).
The \textit{ghost} will be used to communicate with a genuine reader and the \textit{leech} will be used communicate with a genuine card (the victim).
The attacker must trick the user into making a legitimate purchase at the \textit{leech} terminal, and have the devices set up in such a way that they operate at a much larger distance then the normal operational distance of 10 centimeters, so that the attack will go unnoticed. \cite{1128470}

For example, with this attack it could be possible to gain access to public transport.
The check-in gate will communicate with the ghost, which relay the communication on to the leech, which will then relay it to the victim and vice versa.
The check-in gate will conclude it is communicating with the victim, but because the attacker is closer to the gate than the victim, he will gain access first and the victim will pay.

\subsection{Countermeasures against relay attacks}
There are three possible countermeasures: a Faraday cage, interactive confirmation of a transaction by the user or two-factor authentication.
A Faraday cage \footnote{A Faraday cage is a room or cage made out of a mesh of conducting material e.g. copper. This will block out RF signals with wavelengths larger than the mesh spacing.}  will block all RF communication, so an attacker can not make contact with the device.
% TODO faraday cage uitleggen - explain when connectivity is lost
%TODO beetje herschrijven voor duidelijkheid
This is not a feasible countermeasure for a mobile phone, because it will not have any connectivity.

The other alternatives effect a relatively slight decrease in usability, namely because there will be an extra step for a user to switch on NFC features.
Interactive confirmation to activate the connection will prevent the user from accidentally making a transaction.
%The user will have to confirm NFC on the device, to be able to e.g. make a payment or use public transport.
By implementing this, it will not be possible for an attacker to stealthily communicate with a target phone at all times, because users will actually be notified  of all connections to be established.

With two-factor authentication, the user will be asked to present something they know to uniquely identify them (e.g. a password or PIN code, or a shared secret that is agreed upon verbally between users).
%NFC can be turned on at all times, but the user will need to type the secret in order to make use of NFC features.
If the user is aware that the NFC connection is not necessary (because he is not near a payment terminal) and declines or does not type the password this attack can prevented.
%This countermeasure will also decrease user friendliness.
\cite{1128470}

Overall the relay attack on NFC will have lower feasibility than relay attacks on RFID cards, precisely because the user can be presented with the option of accepting or declining transactions.

Another countermeasure against a relay attack is the use of a distance bounding protocol.
Using a distance bounding protocol it is possible for an NFC-enabled mobile device (\textit{Verifier}), to check if the payment terminal or access gate (\textit{Prover}) is within a set distance.
To achieve this, the Verifier sends a challenge to the Prover which has to solve the challenge before the Verifier will allow it to engage in further communication.
This countermeasure works on the principle that any latency incurred in transmitting the challenge to a Verifier must not exceed a certain threshold, ensuring that both communicating parties are within the allowed range of each other.

%The Prover processes the challenge and replies back to the Verifier.
The Verifier will measure the round-trip time between the sending of the challenge and the reception of a valid reply from the Prover.
By subtracting a typical Prover's processing time from the round-trip time, the distance between the two devices can be roughly determined.
This means that the Prover can not easily fake being closer to the Verifier than it really is.
%For NFC this will ideally be 10 centimeters, as this is the specified operational distance. \cite{rasmussenrealization,brands1994distance}
% TODO the specified operational distance - where is this specified? A standard? Which standard?
%TODO q: how does the Verifier know the processing time of the Prover?

%plaatje relay attack

 %2 verschillende scenario's relay attack
 %I - stiekem met iemand z'n kaart praten
 %II - een frauduleus/gemhackte terminal neerzetten waar iemand - willens en wetens - z'n kaart tegenaan houdt.

\subsection{Malware distribution}
Smart poster technology can be used to distribute malware.
It has been proven to be possible to mislead a user to visit a malicious website through a modified RFID-tag using a technique called \textit{Smart poster URI spoofing} which exploits a vulnerability in a user's browser.
To exploit such a vulnerability, an attacker would target a user known to be running a vulnerable browser with a specially crafted message which contains a malicious \textit{Uniform Resource Identifier} (URI) that is disguised to look innocent.
By using whitespace characters like space, tab and new line characters in the transmitted URI it can be made to appear to identify a different resource than it actually does when presented to the user.
The URI addressing malicious content is therefore not shown to user, but the user is still directed there upon visiting the innocent-looking link communicated by the smart poster.
\cite{mulliner2009vulnerability}

%The user is unaware that he or she is directed to a malicious website, from where a number of further attacks are possible.
%In the same way it is possible to let the user call or send a text message to a different number, e.g. an 0900-number.
The different services a user can be directed to through URI spoofing depend on the platform under attack, but could include the World Wide Web, file sharing, instant messaging, (internet) telephony or text messaging.

It is also possible to hide and spread a computer worm which propagates via NFC.
%For such a worm to propagate, the phone of the user has to be infected with the worm.
When a phone infected by such a worm is in proximity to an RFID tag it will attempt to overwrite the data contained in the tag with a malicious URL where a copy of the worm is located, likely disguised as benign software.
If this effort was successful, users that read the tag after its infection will be directed to said website where they will be tricked into downloading and installing a copy of the worm. \cite{rieback2006your}

\subsection{Countermeasures against malware distribution}
This vector is similar to other networking attacks, the best defense against these kinds of attacks is user vigilance in spotting malicious messages.
These attacks can possibly be mitigated by taking preventative measures in the client software.
%TODO reference!

\subsection{Differential Analysis Attacks}
Differential analysis attacks are a class of side channel attack where signal fluctuations (like power usage and electromagnetic radiation) in the system are collected and statistically analysed to yield information about the key or plaintext.

A side channel attack is based on information that can be collected by analysing the physical operation of a cryptographic system (cryptosystem) e.g. by analysing the power usage, electromagnetic radiation leakage or timing of a cryptosystem.
It is possible to prevent this from happening by implementing random behaviour to skew measurements, adding extra shielding or filtering inputs and outputs.

DPA attacks are a common problem in devices where the cryptosystem is designed to be small, cheap and have low power usage (like the Secure Elements used in NFC).
With DPA the power consumption is measured to extract information about the workings of the cryptosystem.
%To measure the power consumption an analog to digital converter is used.
When the cryptosystem is processing, the power usage will vary during the different operations.
This information can be statistically analyzed to determine the encryption key or plaintext, or it can give information about the key which can be used to aid a different attack (such as a statistical brute-force attack). %TODO which attack?

Differential analysis of signals can also be applied to electromagnetic radiation, which is called \textit{Differential Frequency Analysis} (DFA).
Information about the cryptosystem can be collected from the RF noise given off by the electronic signals representing the data travelling to and from the Secure Element and memory.

Many measurement of these variations might necessary to produce the entire key because of noise (e.g. inaccuracies, outside influence, intentional random fluctuations) that is collected during measurements.% which will provide no information about the key.
%Therefore several measurements are necessary to distinguish the signal from the noise.
By aligning the measurements, an attacker can compare data on single points of interest.
By averaging the collected data, the usable signal can amplified and the noise is filtered out.

Other kinds of statistical information about the cryptosystem can be collected by techniques called \textit{fuzzing} and \textit{glitching}, where respectively some bits of input are changed or the function of the device is otherwise tampered with, in order to detect statistical anomalies by correlating these changes with changes measured from the cryptosystem's output(s).
These techniques can also be repeated and the resulting data statistically analyzed to get the key.
%Also, by resetting or switching off the power during a specific operation which can be repeated, information about the key might be collected.
%TODO ref fuzzing glitching

\cite{kocher2009differential}
%TODO ugly sentence, rewrite

\subsection{Countermeasures against Differential Power Analysis}
A DPA attack is successful if enough information can be collected to retrieve the key in the cryptosystem to retrieve the plaintext.
By increasing the number of required measurements to be performed by the attacker, it will take more time to collect the information needed to retrieve the key. 
There are several ways to increase the number of measurements required.
For example, randomly generated noise can be added to the operations of the cryptosystem, which will make it harder to filter out the signal from the noise, causing more measurements to be needed.
Another way is to reduce the signal size, making it harder to pick up. 
If the cryptosystem is designed to self-destruct after a certain amount of operations and the number of measurements needed to retrieve the key is higher than the allowed number of operations, this adds another level of defense against an attacker retrieving the key.

Another countermeasure to prevent information from being collected is clock skipping.
Normally a cryptosystem uses an external clock for its processing.
With clock skipping an internal clock is used in the cryptosystem.
This will create more noise and will prevent the attacker from aligning the points of interest in the collected data and making it more difficult to identify the signal. 
%TODO why is that so?

A countermeasure that can also be effective against DPA is, to use a random number generator to add an amount of unpredictability in the order in which the crypto operations are executed.
The result is that it will be harder for the attacker to find the points of interest in the collected data, making it more difficult to extract the key. \cite{kocher2009differential}

% ERIk open platform vs gesloten platform

\subsection{Denial of Service}
A denial of service attack is also possible, with the goal to render the service unusable. By using a tag that sends out malformed messages, taped on top of the original tag, the phone of the user will crash when he attempts to read the tag.
By putting a copper wire or mesh sticker on top of a tag or a terminal, it is possible to create a small Faraday cage, preventing the terminal to send out its RF signal or reducing the range of the signal. \cite{rieback2006your}

\subsection{Countermeasures Denial of Service}
There are not many countermeasures that you can take to prevent an attack like this. As suggested before, preventive measures taken in client software might help dealing with malformed messages. 
As for Faraday cages in sticker form, only regular checking of the original equipment used, will reduce the impact.

%TODO Countermeasures
