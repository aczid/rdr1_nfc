\chapter{NFC Applications}
\label{chap:applications}
% Hoe hebben we deze lijst gevonden. Is het bedoeld exhaustive te zijn? Nee, alleen de belangrijkste applications worden genoemd.

While NFC is an emerging technology, there are some obvious areas in which it is likely to be applied. Mentioned below are a few applications in which NFC can be used and acts an in illustration to the possiblities of NFC.
% Dit is slechts een kleine greep en bedoeld ter illustratie

\section{Payment} 
Japanese telecom provider \textit{Mobile Felica} has initiated wide deployment of NFC technology in Japan, where virtually every issuer of payment cards supports the \textit{Felica Mobile Wallet}.
In these applications the mobile device emulates the behaviour of an RFID card, but much more is possible by making use of the active communication possibilities of NFC. % TODO much more, zoals? MARK: Geen bron voor felica? Of waar komt dit vandaan?

% Referenties voor maken
In \cite{1555846} the authors investigate the security and usability, and as a whole the feasibility of NFC technology as a replacement for cash.
The authors concluded that their system is "highly usable and is even faster than cash under various common scenarios".

%In "Offline payments with Electronic Vouchers", the authors investigate the technical challenges of a system using cryptographically secure vouchers which can be exchanged between beneficiaries.
%They concluded the biggest limitation....
%MARK het bovenstaande werd weggelaten of wat was hier ook de bedoeling weer van?

\section{Public Transport}
Another application for NFC is public transport. This way it is possible to use a mobile device to pay and gain access to i.e. a train, bus, or metro. By holding your mobile device in front of a terminal you'll pay for a ticket or show your perscription and a gate will open allowing you to enter. The OV-chipkaart in the Netherlands uses this principle, it hasn't been implemented on a mobile device yet, but plans are being made. A mobile device has some advantages above a card, i.e. you can check the balance on your mobile device and increase it inmediately. %MARK niet zo lekker stukje kan een bron bij, maar wat voor IEEE, ACM, scholar? Nog geen pilot in nederland :(

% Vermijd term TC, heeft beladen betekenis MARK: Dus vervangen door? 
\section{Trusted computing}
Strongly related to the progression of payment applications towards 'electronic cash' is the requirement to run trusted code on a user's machine.
This poses several problems as the mobile devices are liable to tampering by the user.
This problem can be mitigated by making use of a \textit{Secure Element}, which can be part of the \textit{Subscriber Identify Module} (SIM) card in users mobile phone (see Chapter 3).
% In de telefoons die 'we' hebben is de SE geen onderdeel van de SIM, maar een los (al dan niet trusted) component. MARK: zullen we dit verder uitbreiden in het architectuur hoofdstuk en dit hier niet noemen?

Because of its sheer microscopic scale, it's very difficult and costly for an attacker to tamper with the function of this device.
%TODO Referenties
In \cite{1497411} the authors investigate the possibility of running trusted code in mobile devices and concluded that while such a system is technically feasible, its widespread adoption is hampered by the certification process payment card companies impose on their payment products.
Payment card companies' current standards dictate that their cards may not be modified after production, which poses a problem as this is exactly what makes NFC an enticing alternative to conventional bank cards.

\section{Healthcare}
Another application for NFC is the healthcare system. In \cite{RFIDHB} an application of NFC is used to meassure the pressure in the eye. A look at Google Scholar gave us the impression that the use of NFC in the healthcare system, is still in the developement stage. Therefore we will leave this application out of our research and recommend it for future work. %Mark niet zo sterk bewijs


\section{Implementation} %goede titel nodig?

As mentioned above, there are different NFC applications (of which payment and public transport are the most important) and these applications are starting to be implemented. There are several reasons why NFC hasn't been implemented yet. First of all, the mobile devices that use NFC, still have to be made available to the consumer, but the consumer will only be interested if he or she can actually do something with NFC. So applications are needed, but therefore different parties have to work together. If i.e. a SIM-card is used as a secure element (see Chapter 3), a bank will need to trust the telecom provider and vice versa, but a bank doesn't like to use a, in their eyes, unsafe SIM-card and a telecom provider doesn't want someone else to mess with their SIM-card. Also the costs involved with trying to get all the shops using NFC, are quite large, because new payterminals have to be purchased.
So all parties have to participate and create agreements to get NFC off the ground, which is what they have been doing, only this takes time. %niet zo sterke zin en gebaseerd op de meeting met erik


% Kip & ei probleem
% Genoeg NFC handsets <-> apps
% trust: telco <-> bank <-> telefoonmaker
% kosten voor payment -> nfc reader bij alle kassa's
% iedereen moet mee doen

% MARK weet niet zo goed wat ik met het onderstaande commentaar moet doen

% SIM - 4k ram 16k rom 64k eeprom

%            trust
% security <             afluister
%            aanvaller < relay attacks
%                        malware
