\section{Introduction}
%\newglossaryentry{PKI}{name={Public Key Infrastructure},description={A cryptographic methode for encrypting messages over untrusted networks.}}
%\glsadd{PKI}

%\section*{Purpose}

%\subsection{Near Field Communication} %(beetje dubbel)
\textit{Near Field Communication} (NFC) is a wireless communication technology which will allow for new applications on mobile devices, e.g. smartphones\footnote{A smartphone is a mobile phone which has other functionalities besides calling and text-messaging, like internet connectivity, email, agenda and the possibility to install third-party software.}.

%NFC is an extension of the \textit{Radio Frequency Identification} (RFID) protocol.
%TODO misschien nog ff RFID algemeen uitleggen (wat een tag/reader is)

%RFID tags can be passive, semi-active or active.
%Active tags contain a battery and transmit at a regular interval whereas semi-active tags can only send data after they have been activated by a reader.
%Passive tags are powered through an electromagnetic field created by the reader.
%Passive tags do not have batteries and can only communicate with a reader when it is within range (ten centimeters according to specifications, although ranges up to ten meters can be covered by more powerful, specialistic equipent).
%Active tags can be read at a range up to about 100 meters.
 
NFC is an extension of the RFID protocol, which means it allows mobile devices to assume the capabilities of an RFID tag or reader, or utilize an RF connection interactively\footnote{By interactive usage, we mean that both parties act as both the sender and the receiver.}.
RFID tags consist of an integrated circuit (microchip) which stores and processes data (e.g. name, fingerprint or product number) and can transmit and receive information wirelessly. %via a modulated radio frequency (RF) signal through an antenna.
These tiny chips can be embedded in products such as bar code stickers, public transport tickets and passports.

NFC enables the creation of a \textit{Wireless Personal Area Network} (WPAN) between mobile handsets and RFID hardware.
This will allow a mobile device to communicate wirelessly with other NFC-capable devices such as the user's own RFID cards, another smartphone, a payment terminal, a so-called `smart poster' from which users can receive extra information or an access control terminal.
Having NFC functionality available in a mobile handset would allow for use cases including payment, electronic banking, access control, exchange of digital coupons or interactive advertising.

RFID-based applications for access control and public transport ticketing systems are already in place, and are likely to make the transition to NFC-enabled mobile phones.
In our opinion, this transition warrants some security research.
NFC systems require security features (similar to those relied upon in RFID technology) in order to qualify as a trusted platform which meets the security criteria for applications where assets of value are exchanged.
The required features include the ability to run signed code which can be remotely updated.
A hardware component which provides these features is called a \textit{Secure Element} (SE) and lies at the core of the security architecture that NFC applications build on.
Access to the SE is regulated by a \textit{Trusted Service Manager} (TSM) who holds the private key necessary to update applications running on it.

%\subsection{Industry adoption of NFC}
Running trusted code on mobile devices has been deemed feasible, but widespread adoption of the technology is hampered by the limitations the payment card industry and telecom providers impose on their products.
The payment card industry's current standards dictate that their cards may not be modified after production, and \textit{Mobile Network Operators} (MNOs) have similar provisions regarding the SIM cards they issue.
%This poses a problem as this potential sharing of resources on the SE between different providers is exactly what makes NFC an enticing alternative to conventional bank cards. \cite{1497411}

%This poses a problem for the implementation of NFC in general.
%\textbf{TODO}
%As seen in (figure xx, still to add) the push for NFC has put some stakeholders in a circular dependency.
%If the customer doesn't demand NFC from a service provider, this party won't request a TSM to create an application, so nothing will be put on a UICC by the MNO and the customer gets nothing.
%If one of the parties in this figure, refuses to cooperate, NFC will not launch.
%What we see now (artikel tweakers en global platform) is that issuers and MNOs are working together to make NFC available for customers.
%The next thing to happen is that mobile phones will get NFC hardware and an implementation of a secure element.
%With Nokia's 6131 NFC phone a first step was made for developers.
%But now also Google has announced that they will introduce a NFC phone and some rumours suggest that also Apple is developing a new Iphone with NFC, see \cite{nu_artikel}.
%If customers like NFC because of the possibilities it will take off and demands for other applications will be made. 

%These SEs will allow for new applications to be built on top of existing RFID technology, as electronic payment and public transport companies are already demonstrating.

%There are two major use cases that sparked its development, namely mobile payments and access control. %MARK we zouden access control toch niet meer gebruiken?

%For users of a mobile device NFC will increase the ease of use, because the user will be able to use a mobile device for paying (e.g. at a shop or candy machine) and use the same device to pay for public transport.
%Theoretically the user doesn't have to carry a wallet anymore, only the mobile device.
%For example, instead of using coins to pay for candy, you'll just wave your smartphone in front of the machine, you'll pay for the candy and the money will be taken out of your bank account. 

%plaatje van een NFC betaling

%TODO Telco's en gebruikers er bij
%TODO Manufacturers of what? - telefoons, NFC tags, ???
%\section{Stakeholders} % was dit niet overbodig om te noemen? MARK: volgens mij zei erik of alles noemen of eruit?
%NFC will be exploited by different stakeholders of the NFC Forum, which was started in 2004 to advance NFC technology.
%Stakeholders are hardware component manufacturers, mobile device assembling companies, SIM card producers, application developers and financial service institutions.
%And of course, let's not forget the user.
%Currently the Forum has 140 members, who all work together to promote the use of NFC. 
%Recently a few pilots have started in the Netherlands which allow users to pay with their mobile device and gain access.
%(misschien hier het voorbeeld noemen van OV chipkaart op de telefoon van Erik Poll).

%\subsection{Implementation}

\subsection{Motivation}
%A lot of literature on the subject is already available.
%To our knowledge there has been no succinct, definitive overview leading from the system organisation to the security architecture and the known vulnerabilities of NFC systems.
Our aim is to present a succinct overview of current research on NFC technology with an emphasis on the security risks potentially involved, and possible countermeasures.


% "Did you google scholar around for this?"
%The goal of this paper, is to provide the reader with an introduction to NFC security-related research currently ongoing.
%In addition to this we hope to be able to identify specific applications in which we foresee vulnerabilities.
% "Introduction to NFC security _research_"
% vs "Introduction to NFC security _issues/architecture_"
% Naar welk aspect kijken wij? - Verderop in 1.5 geven jullie hier beter antwoord op

%TODO
%\textbf{TODO nalopen}
In this paper we will describe various applications of NFC technology and show how these relate to different NFC architectures.
We will also summarize some of the known vulnerabilities surrounding NFC-applications and investigate what possible countermeasures exist.
We hypothesize that most new vulnerabilities will be found in applications built on top of NFC systems, rather than in the NFC architecture itself.
We envision that vendors might try to apply existing business rules to this new technology while they do not fully understand the risks associated with it, which might result in inadequate security measures being taken.

%This is a placeholder for an overview of the subjects covered in our paper, such as applications, hardware and known vulnerabilities and why we chose to cover these topics.
%Also, probably our research question and hypothesis should go in here.

%\subsection{Overview}
%%In chapter \ref{chap:applications} we will briefly summarize some of the applications NFC will likely be used for.
%In chapter \ref{chap:hardware_architecture} we will present an overview of the hardware, software and communication architecture of a NFC-enabled mobile device.
%%\section{...}
%%Further we will look into the possibilities that NFC offers.
%In chapter \ref{chap:known_vulnerabilities} we will summarize security issues on vulnerabilities in NFC or NFC-like applications.

%Based on the architecture and these vulnerabilities, we present a number of foreseen vulnerabilities and explain some possible countermeasures against these foreseen vulnerabilities in chapter \ref{chap:foreseen_vulnerabilities}.



%Bronnen:
%IEEE artikelen:
%Near Field Communication Network Services
%Vulnerability Analysis and Attacks on NFC-enabled Mobile Phones
