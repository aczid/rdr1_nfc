% !TEX TS-program = pdflatex
% !TEX encoding = UTF-8 Unicode

% This is a simple template for a LaTeX document using the "article" class.
% See "book", "report", "letter" for other types of document.

\documentclass[11pt]{article} % use larger type; default would be 10pt

\usepackage[utf8]{inputenc} % set input encoding (not needed with XeLaTeX)

%%% PAGE DIMENSIONS
\usepackage{geometry} % to change the page dimensions
\geometry{a4paper} % or letterpaper (US) or a5paper or....

\usepackage{graphicx} % support the \includegraphics command and options

%%% PACKAGES
\usepackage{booktabs} % for much better looking tables
\usepackage{array} % for better arrays (eg matrices) in maths
\usepackage{paralist} % very flexible & customisable lists (eg. enumerate/itemize, etc.)
\usepackage{verbatim} % adds environment for commenting out blocks of text & for better verbatim
\usepackage{subfig} % make it possible to include more than one captioned figure/table in a single float
% These packages are all incorporated in the memoir class to one degree or another...

%%% HEADERS & FOOTERS
\usepackage{fancyhdr} % This should be set AFTER setting up the page geometry
\pagestyle{fancy} % options: empty , plain , fancy
\renewcommand{\headrulewidth}{0pt} % customise the layout...
%\lhead{}\chead{}\rhead{}
%\lfoot{}\cfoot{\thepage}\rfoot{}

%%% SECTION TITLE APPEARANCE
%%%\usepackage{sectsty}
%%%\allsectionsfont{\sffamily\mdseries\upshape} % (See the fntguide.pdf for font help)
% (This matches ConTeXt defaults)

%%% ToC (table of contents) APPEARANCE
\usepackage[nottoc,notlof,notlot]{tocbibind} % Put the bibliography in the ToC
\usepackage[titles,subfigure]{tocloft} % Alter the style of the Table of Contents
\renewcommand{\cftsecfont}{\rmfamily\mdseries\upshape}
\renewcommand{\cftsecpagefont}{\rmfamily\mdseries\upshape} % No bold!

%%% END Article customizations

%%% The "real" document content comes below...

\title{Security of NFC-enabled mobile devices}
\author{Mark Vijfvinkel \& Aram Verstegen \\ 4077148 4092368}
\date{} % Activate to display a given date or no date (if empty),
         % otherwise the current date is printed 

\begin{document}
\maketitle

\abstract{

We will conduct a literature study into NFC-enabled mobile devices.

}

\section{Problem}

These days most people have at least two things in their pocket, a mobile device (this can be a mobile phone, PDA or now a days a combination of those two, called a smartphone) and a wallet. In the wallet there is a certain amount of money, either in the form of cash or bankcard, or both. And with this everything is payed, from small micropayments (candy from a machine) to big purchases in a shop.
\\ Recently a few pilots have been started \textbf { [bron noemen]} to combine the wallet and the mobile device into one thing. This is done by taking a mobile device and adding a NFC implementation. NFC stands for Near Field Communication, which allows a device to communicate wirelessly which an other device. A result of this will be several implementations, for example one implementation will allow people to pay with their mobile device and an other will use tickets for publictransport and events. 
\\ While these applications will bring a lot of possibillities, it will also introduce security risks. For example, as mentioned before the communication between the mobile device and a paymentsystem will be wireless. As with any wireless communication, it is possible to capture this traffic, if you have the right equipment. Another interesting aspect, is that a mobile device is in hands of a user, who might start tempering the hardware. It is still unknown to us what these vulnerabilities are and if there are some countermeassures present to prevent them. Therefore we will focus, we will focus on the known and forseen vulnerabilities of NFC, by answering a research question.

\subsection{Research question}

At the end of our research we will try to answer to following research question. 
\\

\noindent What is the security architecture of NFC applications for mobile devices and what are some of the known and forseen vulnerabilities? % Moet nog aangepast
\\

\noindent We are able to give the following hypothesis:
\\

\noindent We think that the most vulnerabilities will be found in the transmission of wireless data and the implementation of secure elements. % Even geen idee hoe ik hier verder kan

\subsubsection{Sub-questions}

In order to answer the main research question, we'll answer the following sub-questions.

\begin{itemize}
\item [-] What is the architecture of NFC-enabled mobile device?

\item [-] What are some of the known vulnerabilities in NFC or NFC-like applications?

\item [-] Based on the architecture and known vulnerabilities, what will the forseen vulnerabilities be?

\item [-] How do these vulnerabilities work?

\item [-] What might be possible countermeassures against these vulnerabilities?

\end{itemize}

\section{Scope}
We are aware that because of the time restrictions in place, we won't be able to cover the entire breadth and depth of NFC systems.
In some implementations of an NFC system, a back-end is used to handle transactions.
Because the back-end is likely to be different for each implementation, we will scope our research down to the technical details of a generalized NFC interaction.
We will provide an up-to-date introduction on the architecture of an NFC-enabled mobile device, focusing mainly on security-related components.
Using this work as a guideline we will document some known vulnerabilities in NFC applications, examining their technical details.
After we have found vulnerabilities we will try to find countermeassures against them. 

\section{Motivation}

With the current technology it is possible to make everyday jobs or actions easier and more comfortable. It is now possible to take your phone with you, instead of calling from a house, a was the case a few years ago. Instead of paying with coins and bills, we now pay using a bankcard. But in this century, the evolution of technology does not stop, so ideas like paying with your phone comes to mind. These days the development goes so fast, that things will be overlooked. In IT this is mostly security related. 
\\

\noindent In our opnion security issues should always be addressed in every IT project. Especially when the project will be open to the public and a lot of people will depend on it. When NFC-enabled devices take off and a lot of people start using them and the applications, it will be too late to makes significant changes, that will improve security. Therefore we think it's very important to find and document, the possible security issues regarding NFC-enabled devices.

\section{Deliverables}
We hope to present a clear introduction on the state of NFC-enabled devices, citing some notable deployments and prospective uses.
After this groundwork, we will focus on the technical details of the NFC-enabled devices highlighting some of the security vulnerabilities and risks involved.

\section{Strategy}
In our study we want to investigate the architecture of NFC-enabled devices with a focus on the technical details of the security model. We first hope to reach a general, high-level understanding of the system's capabilities and uses, and write a solid introduction before delving into the more technical details. After this high-level introduction which illustrates the context of our research, we will limit our scope to the NFC devices themselves.
Once we have a basic grasp of the workings of an NFC device, we will document the architecture of the security mechanisms in place. We hope to be able to point out some known or possible vulnerabilities in these.
We would like to also include a case study of an NFC system implementation. This could be entirely theoretical but if time permits, this would warrant some practical experimentation.

\section{Time schedule}
\begin{center}
  \begin{tabular}{ | r | l | l || l | }
    \hline
    Week & Date & Activity & Deliverable \\ \hline
    39 &  & Researching NFC &  \\ \hline
    40 & 8-10 & & Introduction chapter \\ \hline
    41 &  & Technical analysis & \\ \hline
    42 &22-10 & & Technical specifications chapter \\ \hline
    43 & & Autumn break & \\ \hline
    44 & & Case studies & \\ \hline
    45 & 12-11& & Case studies chapter\\ \hline
    46 & & Preparing presentation and draft paper & \\ \hline
    48 & & Preparing presentation and draft paper & \\ \hline
    49 & 6-12 & & Draft paper and presentation sheets due \\ \hline
    49 & 7-12 & & Final presentation  \\ \hline
    49 & 9-12 & Supervisor review & \\ \hline
    50 & 17-12 & Feedback due & \\ \hline
    51 & & Christmas break & \\ \hline
    52 & & Christmas break & \\ \hline
    1 & & Revise paper & \\ \hline
    3 & 17-01 & & Revised final paper due \\ \hline
  \end{tabular}
\end{center}

\bibliography{references}
\end{document}
