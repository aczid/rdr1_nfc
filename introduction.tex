\chapter{Introduction}
\newglossaryentry{PKI}{name={Public Key Infrastructure},description={A cryptographic methode for encrypting messages over untrusted networks.}}
\glsadd{PKI}

%\section*{Purpose}

%TODO Twee keer mobile device e.g. smartphone. 1x introduceren daarna placeholder gebruiken
\section{Near Field Communication} %(beetje dubbel)
Near Field Communication (NFC) is an upcoming wireless communication technology which will allow for new applications on mobile devices, e.g. smartphones.
It will allow a mobile device to communicate wirelessly over a short distance with other NFC-capable devices such as another smartphone, a payment terminal, a so-called 'smart poster' or an access control terminal.
There are two major use cases that sparked its development, namely mobile payments and access control.
For users of a mobile device NFC will increase the ease of use, because the user will be able to use a mobile device for paying (e.g. at a shop or candy machine) and use the same device to pay for public transport.
Theoretically the user doesn't have to carry a wallet anymore, only the mobile device.
For example, instead of using coins to pay for candy, you'll just wave your smartphone in front of the machine, you'll pay for the candy and the money will be taken out of your bank account. 

%plaatje van een NFC betaling

%TODO Telco's en gebruikers er bij
%TODO Manufacturers of what? - telefoons, NFC tags, ???
\section{Stakeholders}
NFC will be exploited by different stakeholders of the NFC Forum, which was started in 2004 to advance NFC technology.
Stakeholders are hardware component manufacturers, mobile device assembling companies, SIM card producers, application developers and financial service institutions.
%And of course, let's not forget the user.
Currently the Forum has 140 members, who all work together to promote the use of NFC. 
%Recently a few pilots have started in the Netherlands which allow users to pay with their mobile device and gain access.
%(misschien hier het voorbeeld noemen van OV chipkaart op de telefoon van Erik Poll).

\section{Security research}
While consumers might see these applications as a gadget that will make their lives easier, this development towards contactless payment systems rightfully raises questions from a security perspective.
Because NFC will be used in payment and access control, it should be assumed that attackers will try to exploit this technology for their own gain. 
Research has been done on the security of NFC in several areas, among which:

% TODO Referenties toevoegen
\begin{itemize}
\item Creating secure off-line payment applications
\item Trusted computing using mobile applications which are managed remotely
\item Network attacks against \textit{Wireless Personal Area} (WPAN) \textit{Networks} (e.g \textit{Denial of Service}, \textit{Snooping}, \textit{Man-in-the-middle}, etc.) \footnote{Even though NFC isn't strictly a WPAN system as it only allows for 2 communicating parties, this paper also covers network attacks against NFC.} %WPAN afkorting introduceren
\item Intrusion detection mechanisms for WPANs
\item Mitigations agains privacy issues related to wireless payment systems
\end{itemize}

To our knowledge there has been no succinct, definitive overview leading from the system organisation to the security architecture and the known vulnerabilities of NFC systems.
% "Did you google scholar around for this?"
What we hope to accomplish with this paper is to provide the reader with an introduction to NFC security-related research currently ongoing.
%In addition to this we hope to be able to identify specific applications in which we foresee vulnerabilities.
% "Introduction to NFC security _research_"
% vs "Introduction to NFC security _issues/architecture_"
% Naar welk aspect kijken wij? - Verderop in 1.5 geven jullie hier beter antwoord op

\newpage

\section{Overview}
In chapter \ref{chap:applications} we will briefly summarize some of the applications NFC will likely be used for.
In chapter \ref{chap:hardware_architecture} we will present an overview of the hardware, software and communication architecture of a NFC-enabled mobile device.
%\section{...}
%Further we will look into the possibilities that NFC offers.
In chapter \ref{chap:known_vulnerabilities} we will summarize security issues on vulnerabilities in NFC or NFC-like applications.

Based on the architecture and these vulnerabilities, we present a number of foreseen vulnerabilities and explain some possible countermeasures against these foreseen vulnerabilities in chapter \ref{chap:foreseen_vulnerabilities}.


%Bronnen:
%IEEE artikelen:
%Near Field Communication Network Services
%Vulnerability Analysis and Attacks on NFC-enabled Mobile Phones
