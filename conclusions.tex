\label{chap:conclusions}
\section{Conclusions}

At the start of this research we had the following research question and hypothesis. 

\begin{quote}
What is the security architecture of NFC applications for mobile devices and what are some of the known and foreseen vulnerabilities?
\end{quote}

\begin{quote}
\textit{We think that most new vulnerabilities will be found in applications built on top of NFC systems, rather than in the NFC architecture itself.
We envision vendors might try to apply existing business rules to this new system, even if their understanding of it is oversimplified.}
\end{quote}

During our research we looked at several possible applications for NFC currently being developed and tested.
The most likely applications to initially make use of NFC appear to be payment and public transport.
In these two sectors it is likely that NFC will become available for consumers, because pilots have been started in these sectors and a lot of research has been done into making these applications feasible.
Because of the current state of affairs in the development of NFC applications, we were unable to test this hypothesis thoroughly.

Making use of the UICC to function as a Secure Element as proposed by GlobalPlatform is hampered by the reluctance of Mobile Network Operators to share the resource with other parties.
The network attacks summarized in this paper are practical attacks against SEs that behave similar to RFID tags, and work regardless of the internal architecture employed by the handset for accessing its SE.

%The concept of an electronic currency as explored by \textit{mFerio} seems very promising
%It's possible for the mobile handset to become a high- target
We find it likely that the development of malware for smartphones will be boosted when the deployment of NFC payment applications becomes widespread.
Smartphone devices, which often already play a central role in their user's lives, will actually become directly financially valuable for criminal interests.
%TODO ugly sentence, rewrite

%Chicken egg problem + verschillende nadelen per architectuur + niet duidelijk welke architectuur het wordt.

%The known vulnerabilities are applicable to all of the SE integration architectures mentioned above.
%It does not matter that the secure element is owned by one party or shared by multiple parties.
%It also does not matter how the secure element is implemented in the architecture of the mobile device.
%The attacks mentioned in chapter 4 all apply to the different architectures mentioned in chapter 3. 

%As mentioned in paragraph 4.5 differential power analysis is a problem for devices where the secure element has to be small, cheap and have low powerusage.
